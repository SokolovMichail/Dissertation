\documentclass[14pt]{mmcs_article}
\usepackage[russian]{babel}
\usepackage{amsmath, amsthm, amsfonts, amssymb}
\usepackage{tikz}
\usetikzlibrary{positioning}



\newenvironment{myenumerate}
{ \begin{enumerate}
		\setlength{\itemsep}{0pt}
		\setlength{\parskip}{0pt}
		\setlength{\parsep}{0pt}     }
	{ \end{enumerate}                  } 

%\graphicspath{{images/}}%путь к рисункам

\begin{document}

% Титульные листы
% раскомментировать требуемое
%%см. РЕКОМЕНДАЦИИ ПО ОФОРМЛЕНИЮ
%И ПРЕДСТАВЛЕНИЮ КУРСОВЫХ И ВЫПУСКНЫХ %КВАЛИФИКАЦИОННЫХ РАБОТ СТУДЕНТОВ ИНСТИТУТА %МАТЕМАТИКИ, МЕХАНИКИ И КОМПЬЮТЕРНЫХ НАУК


% ----------------------------------
% Внимание!
% Изменяйте только строки, перед которыми стоят знаки комментариев
% ----------------------------------

\thispagestyle{empty}
\begin{singlespacing}
\begin{center}

МИНОБРНАУКИ РОССИИ\\ [12pt]
Федеральное государственное автономное образовательное\\
учреждение высшего образования\\
<<Южный федеральный университет>>

\vspace{\baselineskip}
Институт математики, механики\\
и компьютерных наук им.~И.\,И.~Воровича

\vspace{\baselineskip}
% Название выпускающей кафедры
Кафедра алгебры и дискретной математики

\vfill
% Фамилия Имя Отчество студента
\textbf{Иванов Сергей Иванович}

\vspace{\baselineskip}
%%НАЗВАНИЕ РАБОТЫ должно полностью соответствовать распоряжению по Институту (для курсовых работ).
{\bf НАЗВАНИЕ РАБОТЫ, \\
РАЗБИТОЕ ПРИ НЕОБХОДИМОСТИ \\
НА НЕСКОЛЬКО СТРОК }

\vspace{15mm}
КУРСОВАЯ РАБОТА\\
по направлению подготовки\\
% указать направление обучения (раскомментируйте нужную строчку)
01.03.02~-- Прикладная математика и информатика
% 01.03.01~-- Математика
% 01.03.03~-- Механика и математическое моделирование 	
% 02.03.02~-- Фундаментальная информатика и информационные технологии


\vspace{10mm}
\textbf{Научный руководитель~--}\\
% указать данные о руководителе
% должность, степень, звание Фамилия Имя Отчество
проф., д.\,ф.-м.\,н. Сергеев Петр Сергеевич

\vspace{20mm}

\noindent
\begin{flushleft}
$\overline{\textrm{оценка (рейтинг)}}$\qquad	$\overline{\textrm{подпись руководителя\vphantom{й}}}$

\end{flushleft}


\vfill
% год!
Ростов-на-Дону -- 2020

\end{center}

\singlespacing
\end{singlespacing}  % для курсовой
%%см. РЕКОМЕНДАЦИИ ПО ОФОРМЛЕНИЮ
%И ПРЕДСТАВЛЕНИЮ КУРСОВЫХ И ВЫПУСКНЫХ %КВАЛИФИКАЦИОННЫХ РАБОТ СТУДЕНТОВ ИНСТИТУТА %МАТЕМАТИКИ, МЕХАНИКИ И КОМПЬЮТЕРНЫХ НАУК


% ----------------------------------
% Внимание!
% Изменяйте только строки, перед которыми стоят знаки комментариев
% ----------------------------------

\thispagestyle{empty}
\begin{singlespacing}
\begin{center}

МИНОБРНАУКИ РОССИИ\\ [12pt]
Федеральное государственное автономное образовательное\\
учреждение высшего образования\\
<<Южный федеральный университет>>

\vspace{\baselineskip}
Институт математики, механики\\
и компьютерных наук им.~И.\,И.~Воровича

\vspace{\baselineskip}
% Название выпускающей кафедры
Кафедра алгебры и дискретной математики

\vfill
% Фамилия Имя Отчество студента
\textbf{Иванов Иван Сергеевич}

\vspace{\baselineskip}
%НАЗВАНИЕ РАБОТЫ должно полностью соответствовать
% приказу по ЮФУ (для выпускных квалификационных работ)
{\bf НАЗВАНИЕ РАБОТЫ, \\
РАЗБИТОЕ ПРИ НЕОБХОДИМОСТИ \\
НА НЕСКОЛЬКО СТРОК }

\vspace{15mm}
ВЫПУСКНАЯ КВАЛИФИКАЦИОННАЯ РАБОТА\\
по направлению подготовки\\
% Направление обучения
% раскомментируйте нужную строчку
02.03.02~-- Фундаментальная информатика и информационные технологии
% 01.03.01~-- Математика
% 01.03.02~-- Прикладная математика и информатика
% 01.03.03~-- Механика и математическое моделирование 	


\vspace{10mm}
\textbf{Научный руководитель~--}\\
% указать данные о руководителе
% должность, степень, звание Фамилия Имя Отчество
проф., д.\,ф.-м.\,н. Сергеев Петр Сергеевич

\vspace{15mm}

\noindent
% указать Фамилию и инициалы 
% заведующего выпускающей кафедры
\begin{flushleft}
Допущено к защите:\\
заведующий кафедрой \underline{\hspace*{65mm}} Сидоров С.\,С.
\end{flushleft}




\vfill
% год!
Ростов-на-Дону -- 2020

\end{center}

\singlespacing
\end{singlespacing} % для работы бакалавра
%см. РЕКОМЕНДАЦИИ ПО ОФОРМЛЕНИЮ
%И ПРЕДСТАВЛЕНИЮ КУРСОВЫХ И ВЫПУСКНЫХ %КВАЛИФИКАЦИОННЫХ РАБОТ СТУДЕНТОВ ИНСТИТУТА %МАТЕМАТИКИ, МЕХАНИКИ И КОМПЬЮТЕРНЫХ НАУК


% ----------------------------------
% Внимание!
% Изменяйте только строки, перед которыми стоят знаки комментариев
% ----------------------------------

\thispagestyle{empty}
\begin{singlespacing} 
\begin{center}

МИНОБРНАУКИ РОССИИ\\ [12pt]
Федеральное государственное автономное образовательное\\
учреждение высшего образования\\
<<Южный федеральный университет>>

\vspace{\baselineskip}
Институт математики, механики\\
и компьютерных наук им.~И.\,И.~Воровича


\vfill
% Фамилия Имя Отчество студента
\textbf{Соколов Михаил Игоревич}

\vspace{15mm}
%НАЗВАНИЕ РАБОТЫ должно полностью соответствовать 
% приказу по ЮФУ (для выпускных квалификационных работ)
{\bf СОЗДАНИЕ МЕТА-РЕКОМЕНДАТЕЛЬНОЙ СИСТЕМЫ \\
ДЛЯ ИНТЕРНЕТ-МАГАЗИНОВ \\
НА ОСНОВЕ СУЩЕСТВУЮЩИХ МОДЕЛЕЙ }

\vspace{15mm}
ВЫПУСКНАЯ КВАЛИФИКАЦИОННАЯ РАБОТА\\
по направлению подготовки\\
% Направление обучения 
02.04.02~-- Фундаментальная Информатика и информационные технологии,\\
направленность программы\\
<<Математическое и программное обеспечение вычислительных машин>>

\vspace{10mm}
\textbf{Научный руководитель~--}\\
% указать данные о руководителе
% должность, степень, звание Фамилия Имя Отчество
доц., к.\,т.\,н. Чердынцева Марина Игорьевна

\vspace{7mm}
\textbf{Рецензент~--}\\
% указать данные о рецензенте
% должность, степень, звание Фамилия Имя Отчество
к.\,т.\,н. Штейнберг Роман Борисович 


\vspace{15mm}

\noindent
% указать Фамилию и инициалы руководителя
% образовательной программы
\begin{flushleft}
Допущено к защите:\\
руководитель \\
образовательной программы \underline{\hspace*{64mm}} Фёдорова Я.\,М.
\end{flushleft}




\vfill
% год!
Ростов-на-Дону -- 2022

\end{center} 

\singlespacing
\end{singlespacing}% для работы магистра

\renewcommand{\contentsname}{Оглавление}

\tableofcontents

%=======================
\newpage
\addcontentsline{toc}{section}{Постановка задачи}

\section*{Постановка задачи}


Цель работы - создание мета-рекомендательной системы, способной адаптироваться и давать рекомендации для любого интернет-магазина. Для достижения цели был сформирован набор рабочих задач:
\begin{enumerate}
	\item Анализ полученных из интернет-магазинов треков активности пользователей и описания продаваемых товаров
	\item Выбор методов рекомендации, подбор типов моделей-рекомендаторов.
	\item Формирование представлений данных для создания и обучения моделей.
	\item Обучение моделей и поиск оптимальных гиперпараметров.
	\item Добавление возможности автоматического пересоздания представлений данных и моделей и переобучения моделей в связи с изменившимися данными и возможности использования оперативной истории без переобучения моделей
	\item Интеграция в существующую платформу интернет-магазинов.
\end{enumerate}

Для решения задачи( по согласованию с заказчиком) был выбран ЯП Python с пакетами numpy v 1.19.5, scikit-learn v 1.0.0, scipy v 1.7.1, implicit v 0.4.4, pandas v 1.3.5 и TensorFlow v 2.4.2, а также использовался NVidia CUDA Toolkit v 10.1 для ускорения обучения определенных нейросетевых моделей с использованием GPU.

В дальнейшем может быть апдейтнем версию TensorFlow до 2.7.*... Но пока нет. 


%=======================
\newpage
\addcontentsline{toc}{section}{Введение}
\section*{Введение}

Дальнейшее повышение эффективности работы субьектов электронной коммерции, на наш взгляд, связано с расширением списков клиентов, получающих доступ к упомянутым субьектам, а также с персонализацией онлайн-маркетинга. По оценкам McKinsey \cite{INTRO:a1}, 35\% выручки Amazon или 75\% Netflix приходится именно на рекомендованные товары и процент этот, вероятно, будет расти. В данном отчете описана попытка создания решения на базе собранной информации интернет-магазинов. 

Задача рекомендательной системы – проинформировать пользователя о товаре, который ему может быть наиболее интересен в данный момент времени. Клиент получает информацию, а сервис зарабатывает на предоставлении качественных услуг. Услуги — это не обязательно прямые продажи предлагаемого товара. Сервис также может зарабатывать на комиссионных или просто увеличивать лояльность пользователей, которая потом выливается в рекламные и иные доходы.

В зависимости от модели бизнеса рекомендации могут быть его основой, как, например, у компании TripAdvisor, а могут быть просто удобным дополнительным сервисом (как, например, в каком-нибудь интернет-магазине одежды), призванным улучшить Customer Experience и сделать навигацию по каталогу более удобной. В данной работе рекомендательная система используется как дополнительный сервис в дополнение к основному сервису(интернет-магазин), в качестве предмета рекомендаций используются товары магазина. 

Эта работа может быть использована в предприятиях электронной коммерции (интернет-магазинах) для быстрого и легкого создания и обновления автоматической рекомендательной системы и соответственно, имеет высокую практическую значимость.
Результаты данной работы апробированы на конференции "Математика. Компьютер. Образование. 2022".

%=======================
\newpage
\section{Входные данные. EDA}\label{dsfs}
Заказчиком были предоставлены экземпляры файлов, выгружаемых с с интернет-магазинов. 
Файл 1 типа - .csv  файл, содержащий описание событий происходящих в интернет-магазине, в дальнейшем - данные о событиях. \\
Всего файл 1 типа содержит 44 типа полей, однако некоторые из них не заполнены во всем датасете. Далее следует список полей и гипотез для каждого поля. Незаполненные поля, а также поля не несущие полезной информации(сервисные хеши и т.д) опущены.
\begin{myenumerate}
	\item datetime. Поле времени начала события. 
	\item userip. Поле, в котором указывается IPv4-адрес, с которого осуществлялся доступ. Гипотеза: можно использовать для генерации рекомендаций по стране пользователя.
	\item userid. Цифробуквенный уникальный ID пользователя в системе, основной способ идентификации пользователя.
	\item useragent. Фрагмент HTML-header, указывающий на используемый браузер. Гипотеза: можно использовать тип браузера для дифференциации положения владельца в обществе. Нельзя. TODO
	\item eventtype. Тип произошедшего события. 
	\begin{myenumerate}
		\item ProductView - события просмотра товара пользователем. 
		\item AddToCart - события добавления пользователем товара в корзину. 
		\item FacetSelection - событие выбора пользователем фильтра показа товаров.  
		\item Search - событие поиска пользователем товара по названию.
		\item AutoComplete - событие автодополнения запроса пользователя. 
		\item ClickOnSearchResult - событие нажатия на результат поисковой выдачи. 
	\end{myenumerate}
	\item usersearchphrase Запрос пользователя. Используется в типах событий AutoComplete, Search. Гипотеза: Может быть использовано для лингвистического анализа запроса.
	\item correctedsearchphrase. Исправленный запрос пользователя. Используется в типе событий Search. 
	\item facetname. Имя фильтра по которому производился поиск. Используется в типе событий Facet Selection. 
	\item facetvalue. Значение фильтра. Используется в типе событий Facet Selection. Гипотеза: может быть использовано при рекомендации, отфильтровывая рекомендации по соответствующему значению. 
	\item productid. Внутренний уникальный ID продукта. 
\end{myenumerate}

Файл 2 типа - .json файл содержащий описания товаров, а также фильтров товаров, представленных в магазине. Вследствие предполагаемой универсальности системы, а также большого количества полей, являющимися пустыми, приведем описание только используемых универсальных полей, и обобщенное описание остальных полей.\\
В данном файле интерес представляют два поля: Items и Facets. Рассмотрим их подробнее. \\
Поле Items:
\begin{myenumerate}
	
	\item CatalogID. Уникальный числовой ID продукта. Совпадает со значением поля productid в событиях типа ProductView.
	\item Name. Имя товара. Гипотеза: может быть использовано для поиска похожих по названию товаров.
 	\item OrdersCount. Количество заказанных товаров. Гипотеза: можно использовать для фильтрации товаров по популярности.
	\item Типы и значения фильтров, отвечающих товару.
\end{myenumerate}

Поле Facets:
\begin{myenumerate}
	
	\item FacetName. Имя фильтра.
	\item TotalHits. Количество раз, который этот фильтр был выбран. Гипотеза: Может быть использовано для определения самых часто выбираемых фильтров.
	\item Values. Список возможных значений, которые может принимать фильтр. Может иметь как категориальный тип, так и числовой тип. 
\end{myenumerate}

После проведенного EDA, было принято решения использовать лишь следующие поля:

Из файлов 1 типа(событий):

\begin{myenumerate}
		\item datetime.
		\item userid. 
		\item productid.
		\item eventtype. Было принято решение для простоты использовать исключительно типы событий ProductView и AddToCart, так как они несут наибольшее количество полезной информации для рекомендательной системы. 
\end{myenumerate}

Из файлов 2 типа(предметов):

\begin{myenumerate}
	\item CatalogID.
	\item Типы и значения фильтров, отвечающих товару
\end{myenumerate}

Кроме того, такой подход позволяет значительно сократить количество используемой ОЗУ и памяти для хранения данных.

Для обработки данных был создан специализированный пайплайн. Он состоит из 6 стадий обработки данных, которые можно представить в виде следующего графа.

\begin{figure}[H]
\begin{tikzpicture}[main/.style = {draw, rectangle},node distance= 1cm,align=left] 
\node[main] (1) {prepare\_actions(1)}; 
\node[main] (2) [below=0.5cm of 1] {prepare\_df\_splits(2)};
\node[main] (3) [below left=0.5cm and -1cm of 2] {prepare\_csr\_splits(3)}; 
\node[main] (4) [below right=0.5cm and -2cm of 2] {prepare\_items\_by\_users(4)}; 
\node[main] (5) [right=1cm of 1] {prepare\_item\_profiles(5)}; 
\node[main] (6) [below right=0.5cm and -2cm of 4] {prepare\_user\_profiles(6)}; 
\draw[->] (1) -- (2);
\draw[->] (2) -- (3);
\draw[->] (2) -- (4);
\draw[->] (4) -- (6);
\draw[->] (5) -- (6);
\end{tikzpicture}
\caption{Граф пайплайна системы}\label{stud:fig:2}
\end{figure}
На каждом этапе формируется одно из представлений данных. Кратко опишем их.

\begin{myenumerate}
	\item Стадия prepare\_actions. На ней формируется временное представление - $pandas$.$dataframe$, содержаший в себе записи о всех событиях типа  ProductView и AddToCart.
	\item Стадия prepare\_df\_splits. На ней формируется основное представление данных, представляющее собой два  $pandas$.$dataframe$ - соответственно train и test. train содержит в себе все события, произошедшие до определенной даты включительно, test - все события, которые произошли после определенной даты. Важно, что test содержит в себе события только тех пользователей, которые входят во множество train. 
	\item Стадия prepare\_csr\_splits. На ней, на основе результатов стадии \\ prepare\_df\_splits формируются два основных представления данных CSR(csr\_train и csr\_test) - разреженные матрицы формата \\$scipy$.$sparse$.$csr\_matrix$, содержащие все взаимодействия пользователей со всеми товарами(c повышающими коэффициентами для событий AddToCart). Так, одно событие AddToCart рассматривается как одно или более событий ProductView - конкретный коэффициент задается в конфигурационном файле системы. Путем экспериментов, был подобран коэффициент = 10.
	\item Стадия prepare\_items\_by\_users. На этой стадии создается словарь взаимодействия userid и productid, что используется для контроля обучения моделей и в стадии prepare\_user\_profiles.
	\item Стадия prepare\_item\_profiles. На этой стадии формируется представление данных ItemProfileStorage - создается таблица(pandas.DataFrame), содержащая профили всех товаров, построенная на извлеченных из .json файла фильтров показа и их значений. Строки таблицы - товары, столбцы - характеристики товаров. 
	\item Стадия prepare\_user\_profiles. На этой стадии формируется представление данных UserProfileStorage - таблица(pandas.DataFrame), содержащая профили всех пользователей, построенная на извлеченных из .json файла фильтров показа и их значений и представлении данных IP. Каждый профиль пользователя представляет собой алгебраическую сумму профилей всех товаров, с которыми он взаимодействовал(точно так же, как в CSR - c повышающими коэффициентами для событий AddToCart). Это основа для собственного подхода, объединяющего принципы Content based подхода и коллаборативной фильтрации.
	
\end{myenumerate}

Основные представления данных, которые будут использованы в дальнейшем - CSR, ItemProfileStorage, UserProfileStorage.

%=======================

\section{Модели}\label{dsfs}
\subsection{Исследование трудов по теме}
Как уже говорилось выше, поставленная задача - создание максимально универсальной рекомендательной системы, подходящей для любого интернет-магазина при условии стандартизированной выгрузки данных.
Предварительно стоит отметить, что есть две категории товаров:
\begin{itemize}
	\item Повторяемые товары. Это те товары-расходники, которые люди покупают часто. К таким относятся, например, продукты питания, бритвенные станки, предметы бытовой химии.
	\item  Неповторяемые товары. Это такие товары, которые редко приобретают повторно. Примеры таких товаров - электроника, бытовая техника, ювелирные украшения.
\end{itemize}
Давайте опишем нашу рекомендательную систему:
\begin{enumerate}
\item Предмет рекомендации. Для данного проекта предметом рекомендации может являться только товар. Ситуация с применением системы для магазинов, использующих услуги не рассматривалась. В силу предполагаемой максимальной универсальности системы, она предполагает одинаковое отношение как к товарам из повторяемой группы, так и к товарам из неповторяемой группы.
\item Цель рекомендации. Здесь цель рекомендации - информирование пользователя о товарах, которые могут ему подойти, и в конечном счете - покупка пользователем дополнительных товаров.
\item Контекст рекомендации. На момент получения рекомендации предполагается, что пользователь смотрит товары и/или находится в корзине.
\item Иcточники рекомендации. Здесь это как общая аудитория магазина, так и схожие по интересам пользователи, в зависимости от подхода.
\item Степень персонализации рекомендации. Здесь в зависимости от подхода используются разные степени персонализации.
\item Алгоритм рекомендации. Об этом будет написано ниже. 
\end{enumerate}

Путем исследования, было установлено, что существует три основных подхода к созданию моделей для решения следующих задач - Summary-based модели, Content-based модели и Collaborative filtering модели. В подходе collaborative filtering также выделяют отдельный субподход - matrix factorization.  Опишем отличительные особенности этих подходов. 

Подход Summary-based. Данный подход является наиболее простым - и, тем не менее, достаточно эффективным. Он основан на неперсонализированной оценке популярности каждого товара, и соответственно, рекомендации пользователю самых популярных товаров.

Подход Collaborative Filtering. Данный класс систем начал активно развиваться в 90-е годы. В рамках подхода рекомендации генерируются на основании интересов других похожих пользователей, таким образом являясь результатом «коллаборации» множества пользователей. Отсюда и  происходит название метода. Этот метод уже является персонализированным - т.е. рекомендации подбираются персонально под каждого пользователя.

Одними из главных субподходов для collaborative filtering является подходы на основе факторизации матриц - так называемые "matrix factorization". В основе такого субподхода лежит матрица товар-клиент - разреженная матрица взаимодействия оценок товаров и пользователей. Каждое значение суть мера заинтересованности пользователя в товаре. 

Подход  Content-based. Данный подход основан на описании товара. В рамках подхода рекомендуются товары, которые наиболее похожи на популярные у пользователя продукты.

\subsection{Проблема холодного старта}
Проблема холодного старта - одна из типичных ситуаций для рекомендательной системы. Заключается она в том, что в момент добавления нового пользователя и товара о нем практически ничего не известно.

\subsection{} 



Далее были созданы одна модель - baseline и 6 моделей машинного обучения. Перечислим их по порядку: TopN, ALS, BPR, AE, IP, UIP, DRN. Выход каждой модели - список товаров, которые необходимо рекомендовать пользователю. \\
Стоит заметить, что процесс генерации и обучения моделей также является полностью автоматизированным, что позволяет переобучать модели без участия ml-инженеров. К сожалению, все модели(за исключением Top30) жестко привязываются к данным, что делает процесс дообучения на новых данных невозможным, вследствие чего используется процесс полного переобучения.
\subsection{TopN}
Самая очевидная и простая модель - любому пользователю предлагается N самых популярных товаров. Данная модель используется в качестве baseline для всех остальных моделей, а также для дополнения списка рекомендаций в случае если другие модели не смогли дать необходимое их число.
\subsection{ALS}
Модель ALS \cite{ALSA1}. Данная модель принадлежит к классу collaborative filtering. \\
Введем отношения, необходимые для работы данной модели. Пусть у нас есть m пользователей и n продуктов, предлагаемые им. Запишем все взаимодействия пользователей и продуктов в матрицу $R_{m \times n}$, где $R_{i,j}$ - количество раз которое пользователь $i$ взаимодействовал(в нашем случае просматривал) товар $j$. Результирующая матрица является разреженной. \\
ALS(Alternating Least Squares) это итеративный процесс факторизации матрицы $R$ на матрицы $U_{m \times k}$ и $V{k \times n}$ такие, что  $R \approx U^TV$. Здесь $k$ означает количество скрытых признаков.\\
Для нахождения матриц $U$, $V$ необходимо решить следующую оптимизационную задачу: 
\begin{equation}
argmin_{U,V} \sum_{(i,j|r_{i,j} != 0)} (r_{i,j} - U_i^TV_j)^2 + \lambda (\sum_i n_{u_i} {\parallel u_i \parallel} ^2 + \sum_j n_{v_j} {\parallel v_j \parallel} ^2) 
\end{equation}
Здесь $\lambda$ - сила регуляризации, $n_{u_i}$ - количество продуктов просмотренных пользователем $i$,  $n_{v_j}$ - количество раз, которые был просмотрен товар $j$. \\
Поочередно фиксируя матрицы $U$ и $V$, мы получаем квадратное уравнение, которое можно решить, и , таким образом, итеративно улучшить качество разложения. \\
Очевидно, что данная модель использует представление данных CSR. В качестве реализации данной модели была взята библиотека implicit \cite{ALSA2} для ЯП Python.
\\

%\subsection{BPR}
%Модель BPR также принадлежит к классу collaborative filtering. \\
%Для работы модели необходимы ровно те же отношения, что и для работы модели ALS. Внутри, впрочем данная %модель использует метод попарного ранжирования отношения пользователя $i$ к продуктам $j1$ и $j2$. 
%Данная модель также использует модель представления CSR.
\subsection{AE}
Модель AE(AutoEncoder) принадлежит к классу collaborative filtering. \\

Для работы модели необходима матрица взаимодействия пользователь-товар. Идея данного подхода заключается в том, чтобы С помощью нейронной сети предпринимается попытка восстановить строку матрицы взаимодействия пользователь-товар.\cite{AEA1} \\

Как известно, автоенкодер представляет собой нейронную сеть, выполняющую 2 преобразования  $encode(x) : Rn \rightarrow Rd$ и $decode(x) : Rd \rightarrow Rn$. Цель этих преобразований получить представление данных исходной размерности $n$ в размерности $d$ такую, чтобы минимизировать отклонение $x\_received =decode(encode(x))$.

Данная модель представляет собой достаточно простой автоенкодер. Енкодер представлен одним Dense Layer с размерностью входа(n) равной количеству товаров, и размерностью выхода(Bottleneck) экспериментально подобранной как 256(d). Декодер, соответственно, представлен одним DenseLayer с размерностью входа($d$) 256 и размерностью выхода ($n$) равной количеству товаров. 

Особо стоит отметить процесс выбора функций активации и loss-функции. Вследствие характера обрабатываемых данных(implicit feedback пользователей), нежелательна потеря отрицательных значений после активации, что делает нерациональным использование стандартного в таких случаях ReLU( Rectified Linear Unit). Поэтому в качестве функции используется ELU(Exponential Linear Unit). В дальнейшем будет произведен эксперимент с использованием SELU(Scaled Exponential Linear Unit).\\

В подобных задачах часто используют masked mse (loss оценивается только по не нулевым позициям входного вектора, позволяя сети сколько угодно сильно "ошибаться" по тем позциям, где стояли нули.) в качестве loss-функции.  Но при решении данной задачи этот метод не сработал. Обычный mse отлично справляется и не может занулить все наши рекомендации просто по тому, то автоэнкодер не может дать 100\% точность после разжатия данных.


\subsection{IP}
Модель IP является представителем item-based подхода. \\
Построенные в представлении данных IP профили товаров используются как обучающий dataset для модели sklearn.NearestNeighbours, имеющий в основе алгоритм KD-Tree. Данная модель позволяет как рекомендовать товары, похожие на данный, так и рекомендовать товары на основе истории пользователя.  
В случае рекомендаций товаров на основе истории взаимодействий пользователя, применяется следующий алгоритм:

Правило брать не более чем 3 похожих на текущий товара выведено экспериментально.
\subsection{UIP}
Аналогично, построенные в представлении данных UIP профили пользователей используются как обучающий dataset для модели sklearn.NearestNeighbours, имеющий в основе алгоритм KD-Tree. Данная модель позволяет находить пользователей, похожих на данного, и, соответственно, рекомендовать пользователю товары, являющиеся популярными у похожих пользователей.
Применяется следующий алгоритм:

Аналогично, правило брать не более чем 5 товаров у похожего пользователя выведено экспериментально.
\subsection{DRN}

Данная модель примечательна тем, что она является гибридной, совмещая в себе подходы collaborative filtering и content-based. Также, она примечательна и своей конструкцией - это сиамская нейронная сеть, имеющая структуру, приведенную ниже. Задача этой сети - обучиться отличать вещи которые могут понравиться пользователю. Необычен и процесс обучения - на вход данной модели подаются тройки вида (профиль пользователя, профиль понравившейся ему вещи, профиль не понравившейся ему вещи).

Данная модель использует две подмодели - для создания embedding товаров (Dense Layer размерности 64, ReLU, Dense Layer размерности 32, ReLU) и embedding пользователей (Dense Layer размерности 32, ReLU). Оба embedding передаются в ScoreLayer, где определяется близость embedding пользователя к embedding не понравившейся ему вещи и близость embedding пользователя к embedding понравившейся ему вещи.  Финальный слой -TripletLossLayer - определяет разницу score.

Стоит отметить, что слои TripletLossLayer и ScoreLayer реализованы специально для этой модели.

Из-за сложности и специализированных слоев данная модель имеет низкую скорость обучения, и не показывает высоких результатов. Вследствие этого, будет производится исследование возможности доработки данной модели.

\subsection{Сравнительный анализ моделей}
Для сравнительного анализа моделей была рассчитаны метрики для всех моделей на тестовом датасете, содержащем данные за 2 недели, и сведены в единую таблицу.\\
\begin{tabular}{| l |l| l| l| l|}
	\hline
	Модель & DCG@30 & HappyUsersRatio@5 & IOU@30 &  NAP@30 \\
	\hline
	Top30 & 0.02464 & 0.11504 & 0.02029 & 0.00535 \\
	\hline
	ALS & 0.02949 & 0.10442 & 0.01994 &  0.0861 \\
	\hline
	AE & 0.07552 & 0.25841 & 0.04696 &  0.02894 \\
	\hline
	IP & 0.03612 & 0.12249 & 0.02375 &  0.01038 \\
	\hline
	UIP & 0.5002 & 0.26707 & 0.0207 &  0.02035 \\
	\hline
	DRN & 0.00861 & 0.03363 & 0.0027 & 0.00401 \\
	\hline
	Ансамбль & 0.8889 & 0.26707 & 0.04164 &  0.02504 \\
	\hline
\end{tabular}\\ \\
Основополагающей метрикой здесь является метрика HappyUsersRatio.
Как мы видим, наилучший результат показала модель UIP.

%=======================

\section{Финальные рекомендации}
Для улучшения результатов, был создан автоматический ансамбль, который выбирает лучшую collaborative-filtering модель и лучшую content-based - модель(в данном случае единственную - IP). Рекомендации от обоих моделей формируются в шахматном порядке, повторы исключаются, а товары, которые предсказали обе модели идут в начало. Вследствие того, что в ансамбле присутствуют модели, сходные по характеристикам, а в результате обучения метрики немного меняются в зависимости от данных, этот подход позволяет потенциально всегда выбирать наилучшие модели для заданных магазинов.
Сравним метрики, полученные таким способом, с уже имеющимися:\\
\begin{tabular}{| l |l| l| l| l|}
	\hline
	Модель & DCG@30 & HappyUsersRatio@5 & IOU@30 &  NAP@30 \\
	\hline
	Top30 & 0.02464 & 0.11504 & 0.02029 & 0.00535 \\
	\hline
	ALS & 0.02949 & 0.10442 & 0.01994 &  0.0861 \\
	\hline
	AE & 0.07552 & 0.25841 & 0.04696 &  0.02894 \\
	\hline
	IP & 0.03612 & 0.12249 & 0.02375 &  0.01038 \\
	\hline
	UIP & 0.5002 & 0.26707 & 0.0207 &  0.02035 \\
	\hline
	DRN & 0.00861 & 0.03363 & 0.0027 & 0.00401 \\
	\hline
	Ансамбль & 0.8889 & 0.26707 & 0.04164 &  0.02504 \\
	\hline
\end{tabular}\\
\\
Следует отметить, что модели AE и UIP имеет одни из лучших метрик, а итоговый ансамбль опережает даже их, что говорит о целесообразности данного подхода.

%=======================

%=======================
\newpage
\addcontentsline{toc}{section}{Заключение}
\section*{Заключение}

Как итог, на данный момент создана универсальная рекомендательная система с адаптивным ансамблем моделей, способная к полностью автоматическому переформированию представлений данных, пересозданию и переобучению моделей, способная работать для любого интернет-магазина при условии предоставления им стандартизированной выгрузки данных. На данный момент производится А/B тестирование на серверах интернет-магазинов для определения действительной эффективности системы. \\
Кроме того, текущий результат работы был опубликован на конференции МКО-2022 в виде тезисов и видеопрезентации доклада. [Приложения 1,2]

Предложения по доработке.
\begin{enumerate}
	\item Добавить возможность разделения часто и редко покупаемых товаров.
\end{enumerate}



%=======================
\newpage

\addcontentsline{toc}{section}{Литература}
\renewcommand{\refname}{\centering \textbf{Литература}}

% БИБЛИОГРАФИЯ

\begin{thebibliography}{0}
\bibitem{stud:b0}
Рекомендации по оформлению
и представлению курсовых
и выпускных квалификационных работ
студентов института математики,
механики и компьютерных наук.~--
Ростов н/Д, 2020.

\bibitem{stud:b1}
Жуков М.\,Ю., Ширяева Е.\,В.
\LaTeXe: искусство набора и вёрстки текстов с~формулами.~-- Ростов н/Д : Изд-во ЮФУ, 2009.

\bibitem{ALS:a1}{
	Gábor Takács, Domonkos Tikk,
	Alternating least squares for personalized ranking,
	DOI: 10.1145/2365952.2365972 
}

\bibitem{ALS:a2}{
	Github reporistory of "implicit" library,
	https://github.com/benfred/implicit,
	обр. 2021-12-28,
}

\bibitem{AE:a1}{
	Oleksii Kuchaiev, Boris Ginsburg,
	Training Deep AutoEncoders for Collaborative Filtering,
	https://arxiv.org/pdf/1708.01715.pdf 
}

\bibitem{DRN:a1}{
	Ali Elkahky, Yang Song, Xiaodong He,
	A Multi-View Deep Learning Approach for Cross Domain User Modeling in Recommendation Systems,
	https://www.microsoft.com/en-us/research/wp-content/uploads/2016/02/frp1159-songA.pdf 
}

\bibitem{INTRO:a1}
{
	https://www.mckinsey.com/industries/retail/our-insights/how-retailers-can-keep-up-with-consumers
}

\end{thebibliography}



\end{document}
% ----------------------------------------------------------------


\lstset{ %
language=Python,                 % выбор языка для подсветки (здесь это С++)
basicstyle=\small\sffamily, % размер и начертание шрифта для подсветки кода
numbers=left,               % где поставить нумерацию строк (слева\справа)
numberstyle=\tiny,           % размер шрифта для номеров строк
stepnumber=1,                   % размер шага между двумя номерами строк
numbersep=5pt,                % как далеко отстоят номера строк от подсвечиваемого кода
backgroundcolor=\color{white}, % цвет фона подсветки - используем \usepackage{color}
showspaces=false,            % показывать или нет пробелы специальными отступами
showstringspaces=false,      % показывать или нет пробелы в строках
showtabs=false,             % показывать или нет табуляцию в строках
frame=single,              % рисовать рамку вокруг кода
tabsize=2,                 % размер табуляции по умолчанию равен 2 пробелам
captionpos=t,              % позиция заголовка вверху [t] или внизу [b]
breaklines=true,           % автоматически переносить строки (да\нет)
breakatwhitespace=false, % переносить строки только если есть пробел
escapeinside={\%*}{*)}   % если нужно добавить комментарии в коде
extendedchars=true,
commentstyle=\color{mygreen},    % comment style
stringstyle=\bf,
commentstyle=\ttfamily\itshape,
keepspaces=true % пробелы между русскими буквами
aboveskip=3mm,
belowskip=3mm

}


\renewcommand\NAT@bibsetnum[1]{\settowidth\labelwidth{\@biblabel{#1}}%
   \setlength{\leftmargin}{\bibindent}\addtolength{\leftmargin}{\dimexpr\labelwidth+\labelsep\relax}%
   \setlength{\itemindent}{-\bibindent+\fivecharsapprox}%
   \setlength{\listparindent}{\itemindent}
\setlength{\itemsep}{\bibsep}\setlength{\parsep}{\z@}%
   \ifNAT@openbib
     \addtolength{\leftmargin}{\bibindent}%
     \setlength{\itemindent}{-\bibindent}%
     \setlength{\listparindent}{\itemindent}%
     \setlength{\parsep}{0pt}%
   \fi
}
\renewcommand{\thesection}{\arabic{section}.}
\renewcommand{\thesubsection}{\arabic{section}.\arabic{subsection}.}
