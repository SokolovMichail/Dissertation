\documentclass[14pt]{mmcs_article}
\usepackage[russian]{babel}
\usepackage{amsmath, amsthm, amsfonts, amssymb}
\usepackage{tikz}
\usetikzlibrary{positioning}
\usepackage[linesnumbered,boxed]{algorithm2e}
\usepackage{multirow}
\usepackage{xurl}


\newenvironment{myenumerate}
{ \begin{enumerate}
		\setlength{\itemsep}{0pt}
		\setlength{\parskip}{0pt}
		\setlength{\parsep}{0pt}     }
	{ \end{enumerate}                  } 

%\graphicspath{{images/}}%путь к рисункам

\begin{document}

% Титульные листы
% раскомментировать требуемое
%%см. РЕКОМЕНДАЦИИ ПО ОФОРМЛЕНИЮ
%И ПРЕДСТАВЛЕНИЮ КУРСОВЫХ И ВЫПУСКНЫХ %КВАЛИФИКАЦИОННЫХ РАБОТ СТУДЕНТОВ ИНСТИТУТА %МАТЕМАТИКИ, МЕХАНИКИ И КОМПЬЮТЕРНЫХ НАУК


% ----------------------------------
% Внимание!
% Изменяйте только строки, перед которыми стоят знаки комментариев
% ----------------------------------

\thispagestyle{empty}
\begin{singlespacing}
\begin{center}

МИНОБРНАУКИ РОССИИ\\ [12pt]
Федеральное государственное автономное образовательное\\
учреждение высшего образования\\
<<Южный федеральный университет>>

\vspace{\baselineskip}
Институт математики, механики\\
и компьютерных наук им.~И.\,И.~Воровича

\vspace{\baselineskip}
% Название выпускающей кафедры
Кафедра алгебры и дискретной математики

\vfill
% Фамилия Имя Отчество студента
\textbf{Иванов Сергей Иванович}

\vspace{\baselineskip}
%%НАЗВАНИЕ РАБОТЫ должно полностью соответствовать распоряжению по Институту (для курсовых работ).
{\bf НАЗВАНИЕ РАБОТЫ, \\
РАЗБИТОЕ ПРИ НЕОБХОДИМОСТИ \\
НА НЕСКОЛЬКО СТРОК }

\vspace{15mm}
КУРСОВАЯ РАБОТА\\
по направлению подготовки\\
% указать направление обучения (раскомментируйте нужную строчку)
01.03.02~-- Прикладная математика и информатика
% 01.03.01~-- Математика
% 01.03.03~-- Механика и математическое моделирование 	
% 02.03.02~-- Фундаментальная информатика и информационные технологии


\vspace{10mm}
\textbf{Научный руководитель~--}\\
% указать данные о руководителе
% должность, степень, звание Фамилия Имя Отчество
проф., д.\,ф.-м.\,н. Сергеев Петр Сергеевич

\vspace{20mm}

\noindent
\begin{flushleft}
$\overline{\textrm{оценка (рейтинг)}}$\qquad	$\overline{\textrm{подпись руководителя\vphantom{й}}}$

\end{flushleft}


\vfill
% год!
Ростов-на-Дону -- 2020

\end{center}

\singlespacing
\end{singlespacing}  % для курсовой
%%см. РЕКОМЕНДАЦИИ ПО ОФОРМЛЕНИЮ
%И ПРЕДСТАВЛЕНИЮ КУРСОВЫХ И ВЫПУСКНЫХ %КВАЛИФИКАЦИОННЫХ РАБОТ СТУДЕНТОВ ИНСТИТУТА %МАТЕМАТИКИ, МЕХАНИКИ И КОМПЬЮТЕРНЫХ НАУК


% ----------------------------------
% Внимание!
% Изменяйте только строки, перед которыми стоят знаки комментариев
% ----------------------------------

\thispagestyle{empty}
\begin{singlespacing}
\begin{center}

МИНОБРНАУКИ РОССИИ\\ [12pt]
Федеральное государственное автономное образовательное\\
учреждение высшего образования\\
<<Южный федеральный университет>>

\vspace{\baselineskip}
Институт математики, механики\\
и компьютерных наук им.~И.\,И.~Воровича

\vspace{\baselineskip}
% Название выпускающей кафедры
Кафедра алгебры и дискретной математики

\vfill
% Фамилия Имя Отчество студента
\textbf{Иванов Иван Сергеевич}

\vspace{\baselineskip}
%НАЗВАНИЕ РАБОТЫ должно полностью соответствовать
% приказу по ЮФУ (для выпускных квалификационных работ)
{\bf НАЗВАНИЕ РАБОТЫ, \\
РАЗБИТОЕ ПРИ НЕОБХОДИМОСТИ \\
НА НЕСКОЛЬКО СТРОК }

\vspace{15mm}
ВЫПУСКНАЯ КВАЛИФИКАЦИОННАЯ РАБОТА\\
по направлению подготовки\\
% Направление обучения
% раскомментируйте нужную строчку
02.03.02~-- Фундаментальная информатика и информационные технологии
% 01.03.01~-- Математика
% 01.03.02~-- Прикладная математика и информатика
% 01.03.03~-- Механика и математическое моделирование 	


\vspace{10mm}
\textbf{Научный руководитель~--}\\
% указать данные о руководителе
% должность, степень, звание Фамилия Имя Отчество
проф., д.\,ф.-м.\,н. Сергеев Петр Сергеевич

\vspace{15mm}

\noindent
% указать Фамилию и инициалы 
% заведующего выпускающей кафедры
\begin{flushleft}
Допущено к защите:\\
заведующий кафедрой \underline{\hspace*{65mm}} Сидоров С.\,С.
\end{flushleft}




\vfill
% год!
Ростов-на-Дону -- 2020

\end{center}

\singlespacing
\end{singlespacing} % для работы бакалавра
%см. РЕКОМЕНДАЦИИ ПО ОФОРМЛЕНИЮ
%И ПРЕДСТАВЛЕНИЮ КУРСОВЫХ И ВЫПУСКНЫХ %КВАЛИФИКАЦИОННЫХ РАБОТ СТУДЕНТОВ ИНСТИТУТА %МАТЕМАТИКИ, МЕХАНИКИ И КОМПЬЮТЕРНЫХ НАУК


% ----------------------------------
% Внимание!
% Изменяйте только строки, перед которыми стоят знаки комментариев
% ----------------------------------

\thispagestyle{empty}
\begin{singlespacing} 
\begin{center}

МИНОБРНАУКИ РОССИИ\\ [12pt]
Федеральное государственное автономное образовательное\\
учреждение высшего образования\\
<<Южный федеральный университет>>

\vspace{\baselineskip}
Институт математики, механики\\
и компьютерных наук им.~И.\,И.~Воровича


\vfill
% Фамилия Имя Отчество студента
\textbf{Соколов Михаил Игоревич}

\vspace{15mm}
%НАЗВАНИЕ РАБОТЫ должно полностью соответствовать 
% приказу по ЮФУ (для выпускных квалификационных работ)
{\bf СОЗДАНИЕ МЕТА-РЕКОМЕНДАТЕЛЬНОЙ СИСТЕМЫ \\
ДЛЯ ИНТЕРНЕТ-МАГАЗИНОВ \\
НА ОСНОВЕ СУЩЕСТВУЮЩИХ МОДЕЛЕЙ }

\vspace{15mm}
ВЫПУСКНАЯ КВАЛИФИКАЦИОННАЯ РАБОТА\\
по направлению подготовки\\
% Направление обучения 
02.04.02~-- Фундаментальная Информатика и информационные технологии,\\
направленность программы\\
<<Математическое и программное обеспечение вычислительных машин>>

\vspace{10mm}
\textbf{Научный руководитель~--}\\
% указать данные о руководителе
% должность, степень, звание Фамилия Имя Отчество
доц., к.\,т.\,н. Чердынцева Марина Игорьевна

\vspace{7mm}
\textbf{Рецензент~--}\\
% указать данные о рецензенте
% должность, степень, звание Фамилия Имя Отчество
к.\,т.\,н. Штейнберг Роман Борисович 


\vspace{15mm}

\noindent
% указать Фамилию и инициалы руководителя
% образовательной программы
\begin{flushleft}
Допущено к защите:\\
руководитель \\
образовательной программы \underline{\hspace*{64mm}} Фёдорова Я.\,М.
\end{flushleft}




\vfill
% год!
Ростов-на-Дону -- 2022

\end{center} 

\singlespacing
\end{singlespacing}% для работы магистра

\renewcommand{\contentsname}{Оглавление}

\tableofcontents

%=======================
\newpage
\addcontentsline{toc}{section}{Постановка задачи}

\section*{Постановка задачи}
% Везде единый нумерованный список. 

Цель работы ~--- создание мета-рекомендательной системы, способной адаптироваться и давать рекомендации для любого интернет-магазина в области повышения качества обслуживания покупателей и эффективности продаж. Для достижения цели был сформирован набор рабочих задач:
\begin{enumerate}
	\item Провести анализ полученных из интернет-магазинов треков активности пользователей и описания продаваемых товаров.
	\item Выполнить выбор методов рекомендации, а также подбор типов моделей-рекомендаторов.
	\item Сформировать набор представлений данных для создания и обучения моделей.
	\item Выполнить обучение моделей и поиск оптимальных гиперпараметров.
	\item Добавить возможности автоматического пересоздания представлений данных и моделей и переобучения моделей в связи с изменившимися данными и возможности использования оперативной истории без переобучения моделей.
	\item Осуществить интеграцию в одну из существующих платформ интернет-магазинов.
\end{enumerate}

Для решения задач (по согласованию с заказчиком) был выбран язык программирования Python с пакетами numpy v 1.19.5, scikit-learn v 1.0.0, scipy v 1.7.1, implicit v 0.4.4, pandas v 1.3.5 и TensorFlow v 2.4.2. Для ускорения обучения определенных нейросетевых моделей с использованием GPU также использовался NVidia CUDA Toolkit v 10.1.
 
%=======================
\newpage
\addcontentsline{toc}{section}{Введение}
\section*{Введение}

Дальнейшее повышение эффективности работы субьектов электронной коммерции связано с расширением списков клиентов, получающих доступ к упомянутым субьектам, а также с персонализацией онлайн-маркетинга. По оценкам McKinsey \cite{INTRO:a1} 35\% выручки Amazon или 75\% Netflix приходится именно на рекомендованные товары и процент этот, вероятно, будет расти. В работе выполнена попытка создания универсальной рекомендательной системы (мета-рекомендательной системы) на базе собранной информации интернет-магазинов. 

Задача рекомендательной системы ~--- проинформировать пользователя о товаре, который ему может быть наиболее интересен в данный момент времени. Клиент получает информацию, а сервис (на котором работает интернет-магазин и установлена рекомендательная система) зарабатывает на предоставлении качественных услуг. Услуги ~--- это не обязательно прямые продажи предлагаемого товара. Магазин также может зарабатывать на комиссионных или просто увеличивать лояльность пользователей, которая потом выливается в рекламные и иные доходы.

В зависимости от модели бизнеса рекомендации могут быть его основой, как, например, у компании TripAdvisor, а могут быть просто удобным дополнительным сервисом (как, например, в каком-нибудь интернет-магазине одежды), призванным улучшить удобство взаимодействия клиента с сервисом (англ. Customer Experience) и сделать навигацию по каталогу более удобной. 

В данной работе рекомендательная система используется как сервис в дополнение к основной платформе(интернет-магазин), в качестве предмета рекомендаций используются товары магазина. 

В работе проводится исследование полученных данных, проанализированы подходы к созданию рекомендательных систем, выбран набор моделей и реализован ансамбль.

%Добавить какие-то общие места. Добавить 1-2 абзаца всей смысловой части. Указать принцип использования, сколько моделей задействовано, как работает общая политика рекомендации. Указать какой вид связи используется 

Эта работа может быть использована в предприятиях электронной коммерции (интернет-магазинах) для быстрого и легкого создания и обновления автоматической рекомендательной системы и соответственно, имеет высокую практическую значимость.

Результаты данной работы апробированы на конференциях "Математика. Компьютер. Образование. 2022" и СИТО 2022.

%=======================
\newpage


\section{Входные данные. Анализ исходных данных}\label{dsfs}
Любая работа, связанная с машинным обучением обязательно начинается с анализа данных(англ. EDA ~--- Explorative Data Analysis).

Заказчиком были предоставлены экземпляры файлов, выгружаемых из интернет-магазинов.  
Один из файлов содержал описание событий происходящих в интернет-магазине, в дальнейшем ~--- данные о событиях. Данный файл был представлен в формате csv. Второй файл был представлен в формате json и содержал описания товаров, а также фильтров товаров, представленных в магазине. 

Файл с данными о событиях содержал 44 типа полей, однако некоторые из них не заполнены во всем датасете. Далее следует список полей и указаны гипотезы для каждого поля. Незаполненные поля, а также поля не несущие полезной информации(сервисные хеши и т.д), опущены.
\begin{myenumerate}
	\item datetime. Поле времени начала события. 
	\item userip. Поле, в котором указывается IPv4-адрес, с которого осуществлялся доступ.
	\item userid. Цифробуквенный уникальный ID пользователя в системе, основной способ идентификации пользователя.
	\item useragent. Фрагмент HTML-header, указывающий на используемый браузер. 
	\item eventtype. Тип произошедшего события: 
	\begin{myenumerate}
		\item ProductView - события просмотра товара пользователем. 
		\item AddToCart - события добавления пользователем товара в корзину. 
		\item FacetSelection - событие выбора пользователем фильтра показа товаров.  
		\item Search - событие поиска пользователем товара по названию.
		\item AutoComplete - событие автодополнения запроса пользователя. 
		\item ClickOnSearchResult - событие нажатия на результат поисковой выдачи. 
	\end{myenumerate}
	\item usersearchphrase. Запрос пользователя. Используется в типах событий AutoComplete, Search.
	\item correctedsearchphrase. Исправленный запрос пользователя. Используется в типе событий Search. 
	\item facetname. Имя фильтра по которому производился поиск. Используется в типе событий Facet Selection. 
	\item facetvalue. Значение фильтра. Используется в типе событий Facet Selection.  
	\item productid. Внутренний уникальный ID продукта. 
\end{myenumerate}

Рассмотрим подробнее файл,содержащий описание товаров. Вследствие предполагаемой универсальности системы, а также большого количества полей, являющимися пустыми, приведем описание только используемых универсальных полей, и обобщенное описание остальных полей.
В данном файле интерес представляют два поля: Items и Facets. Рассмотрим их подробнее. \\
Поле Items:
\begin{myenumerate}
	
	\item CatalogID. Уникальный числовой ID продукта. Совпадает со значением поля productid в событиях типа ProductView.
	\item Name. Имя товара. 
 	\item OrdersCount. Количество заказанных товаров. 
	\item Типы и значения фильтров, отвечающих товару.
\end{myenumerate}
Поле Facets:
\begin{myenumerate}
	
	\item FacetName. Имя фильтра.
	\item TotalHits. Количество раз, которое этот фильтр был выбран. 
	\item Values. Список возможных значений, которые может принимать фильтр. Может иметь как категориальный тип, так и числовой тип. 
\end{myenumerate}

После рассмотрения списка полей разумно рассмотреть несколько гипотез, которые вытекают из выбранных типов полей:
\begin{enumerate}
	\item Поле userip может использоваться для учета страны пользователя в рекомендациях.
	\item Поле usersearchphrase может быть использовано для лингвистического анализа запроса, что возможно позволит улучшить рекомендации.
	\item Поле facetvalue может быть использовано при рекомендации, отфильтровывая рекомендации по соответствующему значению фильтров.
	\item Поле Name может быть использовано для поиска похожих по названию товаров.
	\item Поле OrdersCount может быть использовано для фильтрации товаров по популярности.
	\item Поле TotalHits может быть использовано для определения самых часто выбираемых фильтров.
\end{enumerate}

Особо следует отметить, что эти стандартизированные выгрузки данных не содержат в себе информации об оценках пользователями купленных товаров, то есть в наличии только неявная обратная связь. Подробнее рассмотрим отличия явной и неявной обратной связи.

% вставить ссылку на одну из ALS статей Collaborative Filtering for Implicit Feedback Datasets
\subsection{Явная и неявная обратная связь}
Рекомендательные системы используют два основных типа обратной связи от пользователей. Это явная обратная связь(англ. explicit feedback) и неявная обратная связь(англ. implicit feedback). Приведем примеры каждой из обратных связей \cite{stud:kimfalk1}.


Явная обратная связь - это оценки контента пользователями. Такими оценками могут быть, например, количество звезд, числовые оценки по некоторым шкалам, и даже простое нравится/не нравится, использующееся, допустим, видеохостингом YouTube. \cite{stud:kimfalk1}

Неявная обратная связь - это любые записанные действия пользователя при взаимодействии с контентом. Просмотры товаров, добавление товара в корзину, покупка товара, история покупок, - все это является одним из вариантов implicit feedback. 

Эти два типа связей значительно отличаются друг от друга по ряду характеристик:
\begin{myenumerate}
	\item Количество данных обратной связи. Для явной обратной связи количество собранных данных всегда значительно меньше, чем количество данных для неявной обратной связи. Кроме того, не всегда в системах вообще существуют встроенные механизмы  сбора данных явной обратной связи и/или возможность их создания. 
	\item Негативные оценки. При явной обратной связи всегда существует явная негативная обратная связь(или возможность ее установления). Так, допустим, недовольный товаром человек может поставить в оценке одну звезду из пяти возможных. При неявной обратной связи же практически невозможно установить негативный feedback, любая такая связь сама по себе будет позитивной. Так, допустим, отсутствие записанного взаимодействия пользователя и конкретной части контента может быть как в силу негативного отношения пользователя к данной части контента, так и в силу отсутствия взаимодействия этих пользователей вообще.  
	\item Наличие шума в данных. Когда мы проводим сбор неявной обратной связи, невозможно достоверно установить их предпочтения и истинные мотивы совершаемых ими действий. Так, например, данные о покупке (или просмотре/добавлении в корзину) пользователем продукта далеко не всегда являются следствием хорошего отношения пользователя к продукту. Существует вероятность, что продукт был приобретен в качестве подарка, или что позднее пользователь был разочарован при использовании продукта.
	\item Отличие в смысле полученных данных. Так, при явной обратной связи числовое значение(количество покупок, оценка качества) означает предпочтение одному продукту другого, в то время как при неявной обратной связи это означает лишь частоту взаимодействия. Тем не менее, хотя более высокая частота взаимодействия не всегда отражает предпочтение одного продукта другому, повторяющееся действие с более высокой вероятностью показывает предпочтения пользователя.
	\item Учет дополнительных признаков. При неявной обратной связи необходимо учитывать также доступность каждого конкретного продукта в момент времени, повторяемые покупки и т.д. При явной обратной связи такое не требуется. 
\end{myenumerate}

По причинам, которые будут описаны в следующем разделе, в работе используется только implicit feedback.

\subsection{Принятые решения об использовании}

После проведенного анализа данных было принято решение использовать лишь следующие поля:

Из файла событий:

\begin{myenumerate}
		\item datetime.
		\item userid. 
		\item productid.
		\item eventtype. Было принято решение для простоты использовать исключительно типы событий ProductView и AddToCart, так как они несут наибольшее количество полезной информации для рекомендательной системы. 
\end{myenumerate}

Из файла товаров:

\begin{myenumerate}
	\item CatalogID.
	\item Типы и значения фильтров, отвечающих товару.
\end{myenumerate}

Такой подход позволяет значительно сократить количество используемой ОЗУ и памяти для хранения данных.

Для обработки данных был создан специализированный пайплайн. Он состоит из 6 стадий обработки данных, которые можно представить в виде графа, показанного на рисунке 1. Пайплайн был выполнен с использованием пакета DVC(Data Version Control) v2.5.4.

\begin{figure}[H]
\begin{tikzpicture}[main/.style = {draw, rectangle},node distance= 1cm,align=left] 
\node[main] (1) {prepare\_actions(1)}; 
\node[main] (2) [below=0.5cm of 1] {prepare\_df\_splits(2)};
\node[main] (3) [below left=0.5cm and -1cm of 2] {prepare\_csr\_splits(3)}; 
\node[main] (4) [below right=0.5cm and -2cm of 2] {prepare\_items\_by\_users(4)}; 
\node[main] (5) [right=1cm of 1] {prepare\_item\_profiles(5)}; 
\node[main] (6) [below right=0.5cm and -2cm of 4] {prepare\_user\_profiles(6)}; 
\draw[->] (1) -- (2);
\draw[->] (2) -- (3);
\draw[->] (2) -- (4);
\draw[->] (4) -- (6);
\draw[->] (5) -- (6);
\end{tikzpicture}
\caption{Граф пайплайна обработки данных}\label{stud:fig:2}
\end{figure}
На каждом этапе формируется одно из представлений данных. Кратко опишем их:

\begin{myenumerate}
	\item Стадия prepare\_actions. На ней формируется временное представление ~--- датафрейм (англ. $pandas$.$DataFrame$), содержащий в себе записи о всех событиях типа  ProductView и AddToCart.
	\item Стадия prepare\_df\_splits. На ней формируется основное представление данных ~--- два  датафрейма ~--- соответственно разбиения train и test. Разбиение train содержит в себе все события, произошедшие до определенной даты включительно, разбиение test - все события, которые произошли после определенной даты. Важно, что test содержит в себе события только тех пользователей, которые входят во множество train. 
	\item Стадия prepare\_csr\_splits. На основе результатов предыдущей стадии  (prepare\_df\_splits) формируются два основных представления данных CSR (csr\_train и csr\_test) ~--- разреженные матрицы , созданные с использованием пакета scipy $scipy$.$sparse$.$csr\_matrix$, размерностью \\ $M \times N$, где $M$ - количество пользователей в системе, а $N$ - количество продуктов в системе. Данные матрицы содержат все взаимодействия пользователей со всеми товарами. 
	
	Следует вспомнить, что рассматриваются события двух видов - AddToCart и ProductView. Так как эти события не являются равнозначными, то было принято решение ввести повышающий коэффициент для событий AddToCart, что позволяет одно событие AddToCart рассматривать как одно или более событий ProductView, причем конкретный коэффициент задается в конфигурационном файле системы. Путем экспериментов был подобран коэффициент равный 10.
	\item Стадия prepare\_items\_by\_users. На этой стадии создается словарь взаимодействия пользователей и товаров. Он используется для контроля обучения моделей и в стадии prepare\_user\_profiles.
	\item Стадия prepare\_item\_profiles. На этой стадии формируется представление данных ItemProfileStorage - создается таблица ($pandas$.$DataFrame$), содержащая профили всех товаров, построенная на извлеченных из json файла фильтров показа и их значений. Строки таблицы - товары, столбцы - характеристики товаров. В основном количестве признаки являются категориальными, но в процессе подготовки представления данных преобразуются в бинарные.
	\item Стадия prepare\_user\_profiles. На этой стадии формируется представление данных UserProfileStorage - таблица($pandas$.$DataFrame$), содержащая профили всех пользователей, построенная на извлеченных из json файла фильтров показа и их значений и представлении данных Item Profile Storage. Каждый профиль пользователя представляет собой алгебраическую сумму профилей всех товаров, с которыми он взаимодействовал (точно так же, как в CSR - c повышающими коэффициентами для событий AddToCart). Это основа для собственного подхода, объединяющего принципы Content based подхода и коллаборативной фильтрации.
	
\end{myenumerate}

Основные представления данных, которые будут использованы в дальнейшем - CSR, ItemProfileStorage, UserProfileStorage.

%=======================

\section{Теоретические основы рекомендательных систем}\label{dsfs}
\subsection{Анализ поставленной задачи}
Как уже говорилось, поставленная задача - создание максимально универсальной рекомендательной системы, подходящей для любого интернет-магазина при условии стандартизированной выгрузки данных.
Предварительно стоит отметить, что есть две категории товаров:
\begin{itemize}
	\item Повторяемые товары. Это те товары-расходники, которые люди покупают часто. К таким относятся, например, продукты питания, бритвенные станки, предметы бытовой химии.
	\item  Неповторяемые товары. Это такие товары, которые редко приобретают повторно. Примеры таких товаров - электроника, бытовая техника, ювелирные украшения.
\end{itemize}
Опишем нашу рекомендательную систему \cite{stud:kimfalk1}:
\begin{enumerate}
\item Предмет рекомендации. Для данного проекта предметом рекомендации может являться только товар. Ситуация с применением системы для магазинов, использующих услуги, не рассматривалась. В силу предполагаемой максимальной универсальности системы она предполагает одинаковое отношение как к товарам из повторяемой группы, так и к товарам из неповторяемой группы.
\item Цель рекомендации. Здесь цель рекомендации - информирование пользователя о товарах, которые могут ему подойти, и в конечном счете - покупка пользователем дополнительных товаров.
\item Контекст рекомендации. На момент получения рекомендации предполагается, что пользователь смотрит товары и/или находится в корзине.
\item Иcточники рекомендации. Здесь это как общая аудитория магазина, так и схожие по интересам пользователи, в зависимости от подхода.
\item Степень персонализации рекомендации. Здесь в зависимости от подхода используются разные степени персонализации.
\item Алгоритм рекомендации. Об этом будет написано ниже. 
\end{enumerate}
\subsection{Обзор подходов к созданию рекомендательных моделей}
Было установлено, что существует три основных подхода к созданию моделей для решения следующих задач ~--- модели на основе популярности(англ. Summary-based), модели на основе содержимого(англ. Content-based) и модели на основе коллаборативной фильтрации(англ. Collaborative filtering). В подходе коллаборативной фильтрации также выделяют отдельный субподход - на основе факторизации матриц(англ. Matrix Factorization).  Опишем отличительные особенности этих подходов. 

Подход на основе популярности. Данный подход является наиболее простым и, тем не менее, достаточно эффективным. Он основан на неперсонализированной оценке популярности каждого товара, и соответственно, рекомендации пользователю самых популярных товаров.

Подход на основе содержимого. Данный подход основан на описании товара. Для товаров создаются признаковые описания, представляющие собой векторы категориальных признаков. В рамках подхода рекомендуются товары, которые наиболее похожи на популярные у пользователя продукты.

Коллаборативная фильтрация. Данный класс систем начал активно развиваться в 90-е годы. В рамках подхода рекомендации генерируются на основании интересов других похожих пользователей, таким образом являясь результатом «коллаборации» множества пользователей. Отсюда и  происходит название метода. Этот метод уже является персонализированным, то есть рекомендации подбираются персонально под каждого пользователя.

Одними из главных субподходов для коллаборативной фильтрации являются подходы на основе факторизации матриц. В основе таких субподходов лежит матрица товар-клиент - разреженная матрица взаимодействия оценок товаров и пользователей. Каждое значение суть мера заинтересованности пользователя в товаре. 

Одна из типичных ситуаций для рекомендательной системы ~--- проблема холодного старта. Данная проблема заключается в том, что при условии добавления нового пользователя/товара система не имеет информации о взаимодействии с ним, и он не будет отображаться в неперсонализированных рекомендациях, потому что им не хватает популярности, чтобы войти в статистику продаж, и он не будет появляться в персонализированных рекомендациях, потому что система не знает, как эти элементы связаны с другими. В нашей системе будет использоваться механизм паддинга рекомендаций, если что система, столкнувшаяся с проблемой холодного старта, не сможет предоставить необходимое количество рекомендаций.  Кроме того, предположительно, система будет пересоздаваться и переобучаться каждые 168 часов, что должно далее нивелировать проблему холодного старта. Наконец, модели, использующие подход на основе содержимого, не испытывают проблем с отсутствием информации о взаимодействии пользователь-товар.

\subsection{Метрики оценки моделей}
Следующим шагом в создании мета-рекомендательной системы будет выбор метрик, способных оценить качество работы системы.

Введём общие обозначения для метрик:
\begin{itemize}
	\item $N$  ~--- количество пользователей.
	\item $K$  ~--- количество товаров.
	\item $rel(i)$ ~--- бинарный признак, показывающий, является ли товар на позиции $i$ релевантным.
\end{itemize}


Оценка моделей производилась по набору следующих метрик:
\begin{enumerate}
	\item Метрика NAP@K(Normalized Average Precision at K) является одной из типичных метрик для измерения качества рекомендательной системы \cite{stud:kimfalk1}. Метрика сравнивает N рекомендаций системы и N наиболее релевантных товаров для пользователя, учитывая их позицию. 
	\begin{equation}
		NAP@K = \frac{1}{N} \sum_{j=1}^{N} \frac{1}{K}\sum_{i=1}^{K}rel(i)
	\end{equation}
	\item Метрика NDCG@K(Normalized Distributed Continued Gain at K) также применяется достаточно часто. Она является модификацией Cumulative Gain at K c учетом порядка элементов.
	\begin{equation}
		nDCG@K= \frac{1}{N} \ sum{j=1}^N \sum{i=1}{K} \frac{2^{rel(i)} }{log_2(k+1)} 	
    \end{equation}   
	\item Метрика IOR@K(Intersection Over Recommended at K). Данная метрика предлагается как аналог NAP, с учетом того что порядок представления рекомендаций не имеет значения. Действительно, для пользователя нет особой важности в том, в каком порядке ему представляются рекомендации.
	\begin{equation}
		IOU@K \frac{1}{N} \sum_{j=1}^N  /frac{(\sum_{i=1}^{K} rel(i))}{K} 
	\end{equation} 
	\item Метрика HUR@K(Happy Users Ratio at K). Данная метрика была предложена заказчиком как методика измерения эффективности системы. Идея данной метрики заключается в том, наша рекомендация считается удачной, если множества рекомендаций нашей модели и множество товаров, с которыми пользователь взаимодействовал,  имеют хотя бы одно пересечение. Она имеет интересное практическое значение - здесь K - количество рекомендаций, которые интернет-магазин может поместить в ленту.
	\begin{equation}
	   HUR@K = \frac{1}{N}\sum_{j=1}^{N} max(1,\sum_{i=1}^{K} rel(i)) 		
	\end{equation}
\end{enumerate}

\section{Модели, используемые в системе}

\subsection{Общее описание моделей}
В системе созданы 7 моделей машинного обучения, одна из которых является baseline. Перечислим их по порядку: TopN, ALS, BPR, AE, IP, UIP, DRN. Выход каждой модели - список товаров, которые необходимо рекомендовать пользователю. 

Отметим, что процесс генерации и обучения моделей также является полностью автоматизированным, что позволяет переобучать модели без участия ml-инженеров. Кроме того, все модели имеют унифицированный интерфейс, что в перспективе позволит добавлять другие модели для улучшения качества рекомендаций. К сожалению, часть моделей(все модели кроме ALS, BPR и TopN) жестко привязываются к данным, что делает процесс дообучения на новых данных невозможным, вследствие чего используется процесс полного переобучения. 


\subsection{Модель TopN}
Модель TopN является одной из наиболее частых и простых в как в составлении, так и в использовании моделей и является представителем подхода на основе популярности. \cite{ALS:recsys} Путем парсинга представления данных CSR производится сортировка товаров по количеству взаимодействий, и любому пользователю предлагается N самых популярных товаров, где N задается необходимостями системы. Данная модель используется в качестве baseline для всех остальных моделей, а также для дополнения списка рекомендаций, в случае если другие модели не смогли дать необходимое их число.
\subsection{Модель ALS}
Модель ALS(англ. Alternating Least Squares) \cite{ALS:CFIFD} ~--- одна из классических моделей рекомендательных систем. Данная модель принадлежит к классу моделей коллаборативной фильтрации. \\
Введем отношения, необходимые для работы данной модели.
% Поменять потом m и n местами.
Примем общее количество пользователей на данный момент времени в системе как $n$, общее количество товаров на данный момент времени в системе как $m$. Запишем все взаимодействия пользователей и продуктов в матрицу $R_{n \times m}$, где $r_{u,i}$ - количество раз которое пользователь $u$ взаимодействовал(в нашем случае просматривал или клал в корзину) товар $i$.  Результирующая матрица является разреженной, так как на практике эффективно невозможна ситуация, когда каждый пользователь взаимодействует с каждым товаром, что отражается нулевыми значениями в матрице. Кроме того, как предложено в статье \cite {ALS:rwe}, введем $\mathrm{T}$ - множество $(u, i)$ : $r_{u,i} > 0$ и $\hat{r}_{ui}$ - предсказанное взаимодействие пользователя с системой.

Процесс факторизации матрицы позволяет представить матрицу $R$ как $R \approx P Q^T$, где $P \in  \mathbb {R}^{n\times k}$ - матрица признаков пользователя, а  $Q \in  \mathbb {R}^{m\times k}$ - матрица признаков товаров, причем $k$ - заранее определенная константа. 
Соответственно $p_u \in \mathbb {R}^{k}$ это строка $u$ в матрице $P$, а $q_i \in \mathbb {R}^{k}$ это строка $i$ в матрице $Q$. 

Формула для предсказания  $\hat{r}_{ui}$:
\begin{equation}
	 \hat{r}_{ui} = p_u^T q_i
\end{equation}

В статье \cite{ALS:CFIFD} предложено ввести переменные $s_{ui}$, которые можно принять как меру предпочтения пользователя $u$ товару $i$. Для удобства использования бинаризуем их. Тогда
\begin{equation}
	pr_{ui} = \begin{cases}
		1 & r_{ui} > 0 \\
		0 & r_{ui} = 0
	\end{cases}
\end{equation}

Тем не менее сразу использовать эти значения $pr_{ui}$ нерационально. Авторы статьи \cite{ALS:CFIFD} предлагают введение дополнительных переменных $c_{ui}$ таких, что $c_{ui} = 1 + \alpha * r_{ui}$. Эти переменные отвечают за меру нашей уверенности в значении $pr_{ui}$.  Всегда будет существовать ненулевое предпочтение пользователя $u$ товару $i$, и при этом, чем больше количество взаимодействий пользователя и товара зафиксировано, тем более высокое значение примет $c_{ui}$. Параметр $\alpha$ здесь повышающий коэффициент, который предлагается принять равным 40. 

%TODO Добавить кусочек про модель без оптимизации.

Введем ценовую функцию:
\begin{equation}
	g(P, Q) = \sum_{u, i \in \mathrm{T}} (c_{ui}(\hat{r}_{ui} - r_{ui})^2 + \lambda_P\parallel p_u \parallel^2 + \lambda_Q\parallel q_i \parallel^2) 
\end{equation}

В силу того что будет применена оптимизация путем использования метода сопряженных градиентов, значения $c_{ui}$ и $r_{ui}$ не будут константными в позициях $(u,i) \not\in \mathbb{T}$. Обозначим такие исходные значения как $c_0$, $r_0$.

Особенность примененной здесь модели заключается в том, что для ускорения процесса вычислений применяется взвешенная гребневая регрессия, которую можно обозначить как WRR ~--- Взвешенная гребневая регрессия (англ. Weighted Ridge Regression) \cite{implicit:all}.

Приведем один из алгоритмов, часто применяющихся для приближенного расчета WRR-метода сопряженных градиентов с необходимой точностью $\epsilon$ \cite{ALS:recsys}.

\begin{algorithm}[H]\label{al:1}
	\caption{Предусловленный метод решения системы $Aw = b$ методом сопряженных градиентов.}
	\KwIn{$A, M \in \mathbb{R}^{d \times d}, b, w_0 \in \mathbb{R}^{d \times 1}, E \in \mathbb{N}$}
	\KwOut{$w_0 \in \mathbb{R}^{d}$}
	
	$w \leftarrow w_0, r \leftarrow b - Aw, z \leftarrow M^{-1}r, p \leftarrow z$\\
	\For{$k \leftarrow 1...E$}{
		\eIf{$\parallel r \parallel < \epsilon$}{return $w$}
		{$\gamma$ $\leftarrow$ $r^Tz$
		 $\alpha \leftarrow \gamma / (p^TAp)$
		 $x \leftarrow x + \alpha r$
		 $r \leftarrow r - \alpha Ap$
		 $z \leftarrow M^{-1} r$
		 $\beta \leftarrow \gamma / (r^TAz)$
		 $p \leftarrow z + \beta p$
	 	}
	}
\end{algorithm}

Расчет $(u,i)$ происходит по следующим формулам:
 \begin{equation}
 	\hat{r}_{ui} = p_u^T q_i, \\ 
\end{equation}
\begin{equation}
	 p_u = s_u \sum_{j \in \mathbb{J}}w_j,\\ 
\end{equation}
\begin{equation}
	 s_u = (n_u + 1)^{\frac{1}{2}}
\end{equation} При этом ценовая функция остается такой же, с тем отличием, что $P$ может быть выражена через $W$ и обучающее множество. Ценовая функция является невыпуклой, поэтому используется не точная, но примерная минимизация по методу IALS.

Пусть $W \in \mathbb{R}^{m \times k}$ -матрица векторов $w_j$.  Введем $B \in R^{n \times m}$ такую, что $B_{ui} = r_{ui}s_u$. Исходя из этого, верно следующее выражение: $P = BW$, где $B$ - это разреженная матрица. 

Во время оптимизации используется три комплекта параметров - $P, Q, W$. Процесс оптимизации представлен листингом \ref{al:2}, где WRR - алгоритм \ref{al:1}.\\
\begin{algorithm}[H]\label{al:2}
	\caption{Процесс оптимизации алгоритма ALS c помощью WRR.}
	 \For{$e = 1 ... I$}{
		 $Q-step$: \For{ $i \in \mathbb{I}$}{$q_i \leftarrow WRR(P, \bar{r}_i, \bar{c}_i,n_i\lambda_QI)$}
		 $P-step$: \For { $u \in \mathbb{U}$}{$p_u \leftarrow WRR(Q, r_u, c_u,n_u\lambda_PI)$}
		 $W-step$: \For {$k \in \mathbb{K}$}{$q_i \leftarrow WRR(B, \bar{p}_k, 1 ,\lambda_WI)$}
		 \For {$u \in \mathbb{U}$} {$p_u \leftarrow p_u = s_u\sum_{j \in \mathbb{I_u}} w_j$}
	}
\end{algorithm}
\ \\
Рассмотрим шаги алгоритма подробнее.
Первый шаг оптимизирует $Q$ в уравнении $R \approx PQ^T$. Здесь $\bar r_i \in \mathbb{R}^{N \times 1}$  и  $r_u \in \mathbb{R}^{M \times 1}$ - столбец $i$ и ряд $u$ матрицы $R$. Векторы $ \bar{с_i} \in \mathbb{R}^{N \times 1}$ и $\bar{с_u} \in \mathbb{R}^{M \times 1}$ соответственно составлены из значений $c_ui$. В качестве WRR может быть использован любой алгоритм расчета взвешенной гребневой регрессии, причем $\bar r_i$, $\bar c_i$ не имеют необходимости полного пересчета каждый раз, вследствие того, что изменяется лишь малая их часть. 

Второй шаг очень похож на первый, но оптимизирует лишь $P$. 

Третий шаг отличен в том, что он оптимизирует только $W$ из $P \approx BW$, причем он направлен на лучшую аппроксимацию матрицы $P$, а не $R$. Здесь  $\bar p_k \in \mathbb{R}^{N \times 1}$ ~--- это $k$-й столбец матрицы $P$, веса взвешенной гребневой регрессии константно равны 1.

Описанный алгоритм реализован в библиотеке implicit v.0.4.4, которая используется в данном проекте. Библиотека implicit требует представление данных CSR.


\subsection{Модель BPR}
Модель BPR(англ. Bayesian Personalized Ranking) также принадлежит к классу collaborative filtering. 

Обозначим множество всех пользователей как $U$, множество всех товаров как $I$. Доступные данные по Implicit feedback ~---  $S = U \times I$. Введем функции предпочтения пользователя: $ >_u \in I^2$, которая отвечает следующим условиям:
\begin{itemize}
	\item $\forall i, j : i \neq j \rightarrow i >_u j \vee j >_u i$
	\item $\forall i, j : i >_u j \wedge j >_u i \rightarrow i = j$
	\item $\forall i, j, k :  i >_u j \wedge j >_u k  \rightarrow i >_u k $
\end{itemize}
Также, статья \cite{BPR:1205} предлагает ввести следующие обозначения. 
\begin{itemize}
	\item $I_u+: \{i \in I \wedge (u,i) \in S\}$
	\item $U_i+: \{u \in U \wedge (u,i) \in S\}$
\end{itemize}

Как уже говорилось, зачастую данные Implicit Feedback крайне разреженные. Кроме того, в Implicit Feedback присутствует только положительная обратная связь ~--- остальные данные не заполнены. Стоит еще раз отметить, что implicit feedback ~--- незаполненные данные на самом деле являются смесью отрицательных данных и незаполненных.
Обычно такие данные просто отбрасываются, но в нашем случае они будут использованы.

Типичный подход для рекомендательных систем ~--- расчет такого значения $x_{ui}$, которое отражает степень предпочтения пользователя к данному товару. Тренировочным данным $(u,i) \in S$ присваивается позитивная метка, а любым другим комбинациям $(u,i) \notin S$ ~--- негативная. Это означает, что модель натренирована предсказывать 1 для предметов принадлежащих $S$, и 0 - в противном случае. Однако в данном подходе есть проблема ~--- все элементы, которые модель должна ранжировать в дальнейшем, представлены для обучения модели как негативная обратная связь. Соответственно, достаточно чувствительная модель будет иметь плохую эффективность - ведь она будет предсказывать только 0.    

 Применяется другой подход ~--- используем в качестве тренировочного множества пары товаров, причем оптимизация происходит в зависимости от корректности ранжирования пары, в отличие от использования одиночных "верных" \  товаров \cite{BPR:1205}. То есть задача ~--- насколько это возможно, получить из $S$ отношения $>_u$ для каждого пользователя. Так, если $ (u,i) \in S $ ~--- то можно считать что пользователь $ u $ предпочитает $ i $. Формализуем: создадим тренировочный датасет $ D_S: U \times I \times I $ такой, что:
\begin{equation}
	D_S := \{ (u,i,j) | i \in I_u^+ \wedge j \in I \setminus I_u^+ \}
\end{equation}
То есть мы создаем пары товаров для пользователей такие, что пользователь $ u $ предпочитает товар $ i $ товару $ j $.
Данный подход имеет два преимущества:
\begin{itemize}
	\item Датасет $D$ состоит из позитивных и негативных пар и незаполненных значений. Незаполненные значения между двумя товарами, с которыми не взаимодействовал пользователь, это те значения, которые требуется предсказать. Соответственно, если рассмотреть попарно, то тренировочный и тестовый датасеты разделены.
	\item При этом тренировочные данные являются подмножеством $D$. 
\end{itemize}     

Рассмотрим алгоритм работы BPR \cite{BPR:1205}.
Формулировка Байесовской задачи для нахождения персонализированного ранжирования для товара $  i \in I$ заключается в максимизации следующей условной вероятности:
\begin{equation}
	p(\theta | >_u) \sim p(>_u | \theta)p(\theta).
\end{equation} 
Здесь $\theta$ ~--- вектор параметров модели, а $ >_u $ ~--- это действительные предпочтения пользователя $ u $, причем предполагается, что все пользователи действуют независимо друг от друга. Кроме того, предполагается что порядок каждой пары товаров $ i,j $ является независимым от порядков других пар. Отсюда, функцию $ p(>_u | \theta) $ можно 
переписать как:
\begin{multline}
	\prod_{u \in U} p(>_u | \theta) = 	\prod_{(u,i,j) \in U \times I \times I} p(i >_u j | \theta )^{\sigma((u,i,j) \in D_S)} \cdot \\ \cdot (1 - p(i >_u j | \theta ))^{\sigma((u,i,j) \notin D_S)}
\end{multline}
Причем
\begin{equation}
	\sigma(b) = \begin{cases}
		1 & if \ b == true \\
		0 & otherwise
	\end{cases}
\end{equation}
Однако, используя тот факт, что схема попарного ранжирования всеобща и антисимметрична, можно упростить данную формулу до:
\begin{equation}
	prod_{u \in U} p(>_u | \theta) = \prod_{(u,i,j) \in D_S} p(i >_u j | \theta)
\end{equation}

Чтобы гарантировать исполнение свойств всеобщности и антисимметричности, введем индивидуальную вероятность того, что пользователь $ u $ предпочтет товар $ i $ товару $ j $ как $p(i >_u j) = \sigma(\hat{x}_{uij}(\theta))$, где $\sigma(x) = \frac{1}{1+e^{-x}}$. Здесь $ \hat{x}_{uij}(\theta) $ ~--- функция от вектора $\theta$, которая и описывает отношения пользователя $u$ и товаров $ i,j $. Таким образом, мы делегируем моделирование отношений внутреннему классу модели (здесь использована факторизация матриц, описанная в параграфе про ALS).  

Для того чтобы закончить моделирование по Байесу, введем понятие "Общей плотности" $ p(\theta) $, которая является нормальным распределением с математическим ожиданием = 0 и матрицей ковариации $ \sum_{\theta} $: $p(\theta) = N(0,\sum_{\theta})$.
Сокращая количество неизвестных переменных, представим $ \sum_{\theta} $ как $ \lambda_{\theta} I $. Тогда сформулируем Байесовский критерий оптимизации BPR-Opt:
\begin{equation}
	\begin{split}
		BPR-Opt = ln(p(\theta) |  >_u) = \\
		ln(p(>_u | \theta)p(\theta)	= \\
		ln \prod_{(u,i,j) \in D_S} \sigma_{\hat{x_{uij}}}p(\theta) = \\ 
		\sum_{(u,i,j \in D_S)}ln \ \sigma(\hat{x}_{uij}) + ln \  p(\theta) = \\
		\sum_{(u,i,j \in D_S)}ln \ \sigma(\hat{x}_{uij}) - \lambda_{\theta} \parallel \theta \parallel ^ 2
	\end{split}
\end{equation}

Рассмотрим сам алгоритм обучения LearnBPR. Это специальный алгоритм, основанный на стохастическом градиентном спуске и динамической выборке триплетов для тренировки.

Найдем градиент BPR-Opt.
\begin{multline}
	\frac{\partial BPR-Opt}{\partial \theta} = \sum_{(u,i,j \in D_S)} \frac{\partial}{\partial \theta} ln \ \sigma(\hat{x}_{uij}) - \lambda_{\theta} \frac{\partial}{\partial \theta} \parallel \theta \parallel ^ 2 \propto \\
	 \sum_{(u,i,j \in D_S)} \frac{- e^{-\hat{x}_{uij}}}{1 + e^{-\hat{x}_{uij}}} \frac{\partial}{\partial \theta} \hat{x}_{uij} - \lambda_{\theta} \theta
\end{multline}

Соответственно, можно записать и общий алгоритм LearnBPR.\\
\begin{algorithm}[H]\label{bpr:1}
	\caption{Общий алгоритм LearnBPR.}
	\KwIn{$D_S, \theta$}
	\KwOut{$\theta$}
	initialize $\theta$ \\
	\While{not convergence}{
	take $(u,i,j)$ из $D_S$ \\
		$\theta \leftarrow \theta + \alpha (\frac{- e^{-\hat{x}_{uij}}}{1 + e^{-\hat{x}_{uij}}} \frac{\partial}{\partial \theta} \hat{x}_{uij} - \lambda_{\theta} \theta) $
	}
	return $\theta$
\end{algorithm}

Особо следует отметить выбор данных для тренировки. Если выбирать данные триплетов, принадлежащих к одному пользователю или товару, то сходимость даже стохастического градиентного спуска будет медленной, так как для пары $(u,i)$ существует много $j$, для которых выполняется что $(u,i,j) \in D_S$. В статье \cite{BPR:1205} предложено выбирать триплеты случайно и равномерно. С таким подходом шанс выбрать одинаковые пары "пользователь-товар"\ достаточно мал, и поэтому алгоритм сходится значительно быстрее.

\subsection{ Модель AE}
Модель AE(AutoEncoder) принадлежит к классу collaborative filtering, а также является представителем semi-supervised learning \cite{AE:a1}. 

Автоэнкодер представляет собой нейронную сеть, состоящую из двух частей ~--- энкодер(англ. encoder) и декодер(англ. decoder). Эти части соответственно выполняют преобразования  $encode(x) : R^n \rightarrow R^d$ и $decode(x) : R^d \rightarrow R^n$, где $n$ ~--- исходная размерность данных, а $d$ ~--- целевая размерность. Цель этих преобразований: получить представление данных исходной размерности $n$ в размерности $d$ такое, чтобы минимизировать отклонение исходных данных от полученных: $x - decode(encode(x)) -> min$.

В данной модели и энкодер и декодер ~--- это feed-forward сети с полносвязными слоями $I = f(W*x + b)$, где $f$ ~--- некоторая нелинейная активационная функция, причем архитектура декодера такая же, как и архитектура энкодера. 

Рассмотрим принцип работы систем. Forward pass ~--- на вход подается вектор рейтингов пользователя $x \in R^M$, где $M$ ~--- количество продуктов. На выходе мы получаем вектор $x \in R^M$, но с одним значительным отличием ~--- на входе вектор был разреженным, на выходе все значения пользователя заполнены.

Особо стоит отметить процесс выбора функций активации и loss-функции. Для функции активации классическим решением было бы использование RELU(Rectified Linear Unit). Однако, вследствие характера обрабатываемых данных (неявная обратная связь от пользователей), нежелательна потеря отрицательных значений после активации, что делает нерациональным его использование. Поэтому в качестве функции используется ELU (Exponential Linear Unit). Также был проведен эксперимент с использованием SELU(Scaled Exponental Linear Unit). Эксперименты проводились на трех размерностях бутылочного горлышка(англ.bottleneck) ~--- 128, 256, 512. Результаты внесены в таблицу.\\

Приведем формулы всех трех функций активации.
\begin{equation}
	ELU(x) = \begin{cases}
		x & x > 0 \\
		exp(x) - 1 & x \leq 0
	\end{cases}
\end{equation}
\begin{equation}
	RELU(x) = max(0,x)
\end{equation}
\begin{equation}
	SELU(x) = scale * (max(0,x) + min(0, \alpha * (exp(x) - 1)))
\end{equation}

Применение таких функций дало результаты, представленные в таблице \ref{AE:table:1}.
\begin{table}[H]
\centering
\caption{Результаты использования различных функций активации}\label{AE:table:1}
\begin{tabular}{| l | l |l| l| l| l|}
	\hline
	Размерность & Функция & DCG@30 & HUR@5 & IOU@30 &  NAP@30 \\
	\hline
	\multirow{3}{4em}{128} & ELU &  0.06878 & 0.24956 & 0.04602 & 0.02510 \\
	
	 & SELU & 0.05771 &  0.22301 & 0.03988 & 0.02014 \\
	
	 & RELU &  0.06313 & 0.23894 & 0.03917 & 0.02453 \\
	 \hline
	 \multirow{3}{4em}{256} & ELU & 0.07552 & 0.25841 & 0.04696 &  0.02894 \\
	 
	 & SELU & 0.07021 & 0.24779 & 0.04566 & 0.02497 \\
	 
	 & RELU &   0.06343 & 0.24248 & 0.03988 & 0.02427 \\
	 \hline
	 \multirow{3}{4em}{512} & ELU &  0.07842 & 0.25664 & 0.04773 & 0.03082\\
	
	 & SELU & 0.07398 & 0.26018 & 0.04678 & 0.02780 \\
	 
	 & RELU & 0.06468 & 0.24071 & 0.03947 & 0.02510 \\
	 \hline

\end{tabular}

\end{table}


В подобных задачах \cite{AE:a1} часто используют MMSE(Masked Mean Squared Error) (loss оценивается только по не нулевым позициям входного вектора, позволяя сети сколько угодно сильно "ошибаться"\ по тем позициям, где стояли нули) в качестве loss-функции.
\begin{equation}
	MMSE = \frac{m_i * (r_i - y_i) ^ 2}
	{\sum_{i=0}^{i=n} m_i}
\end{equation}  
Однако в процессе экспериментов выяснилось, что результаты нейронной сети с такой конфигурацией значительно хуже. Тогда было принято решение использовать обычный MSE(Mean Squared Error), имеющий формулу:
\begin{equation}
	MSE = \frac{1}{n} \sum_{i=1}^n(r_i-y_i)^2
\end{equation}
MSE не может занулить все наши рекомендации просто потому, что автоэнкодер не может дать 100\% точность после разжатия данных.

Модель реализована с использованием пакета TensorFlow 2.4.2. Использование программного пакета CUDA позволяет производить обучение с использованием графических ускорителей. Обучение модели происходит на протяжении 500 эпох, с использованием batch\_size = 256.Время обучение даже с использованием графического ускорителя достаточно велико ~--- около 30 минут. Архитектура модели представлена на рисунке \ref{AE:pic:1}.

 

%TODO Check
\begin{figure}[H]
	\begin{tikzpicture}[main/.style = {draw, rectangle},node distance= 1cm,align=left]
		\node[main] (1) {Input(, N)}; 
		\node[main] (2) [right=1cm of 1] {Dense(256)}; 
		\node[main] (3) [right=1cm of 2] {ELU};
		\node[main] (4) [right=1cm of 3] {Dense(256)}; 
		\node[main] (5) [right=1cm of 4] {ELU};
		\draw[->] (1) -- (2);
		\draw[->] (2) -- (3);
		\draw[->] (3) -- (4);
		\draw[->] (4) -- (5);
	\end{tikzpicture}
	\caption{Архитектура модели AE}\label{AE:pic:1}
\end{figure}


\subsection{Модель IP}
Модель IP(Item Profiles) является представителем подхода на основе содержимого \cite{stud:kimfalk1}. \\
Создается представление данных IP, в котором каждый товар представляется как вектор размерности $n$, где $n$ ~--- общее количество бинаризованных фильтров. То есть существует метрическое пространство $\mathbb{B}^n$ c набором векторов $x \in \mathbb{B}^n$. Задача - для  $\forall q \in \mathbb{R}^n$ найти $N$ ближайших точек к $q$.

Построенные данные используются для обучения модели, основанной на алгоритме нахождения ближайших соседей, который реализован в пакете Sklearn как Sklearn.NearestNeighbours. Рассмотрим его подробнее.

Алгоритм Sklearn.NearestNeighbours автоматически выбирает один из трех алгоритмов ~--- BallTree, KDtree и brute \cite{IP:a1}. Кратко рассмотрим эти алгоритмы:
\begin{enumerate}
	\item KDTree \cite{IP:a2} ~--- бинарное дерево, где каждый узел ~--- точка с размерностью $n$. Можно представить, что каждая точка является гиперплоскостью, разделяющей пространство на два подпространства. Направление гиперплоскости выбирается следующим образом: каждая точка ассоциируется с одной из $n$ осей, а гиперплоскость выбирается ортогонально данной оси;
	\item BallTree ~--- бинарное дерево, где каждый узел определяет гиперсферу с размерностью $n$, которая содержит в себе все точки;
	\item brute ~--- наиболее простой brute-force алгоритм, находящий все возможные расстояния точек друг от друга.
\end{enumerate}

Необходимо выбрать метрику расстояния, которая будет использоваться в алгоритме Sklearn.NearestNeighbours. Примем, что все метрики находят расстояние между двумя одномерными тензорами $v1$ и $v2$ размерности $n$.  Рассмотрим несколько возможных вариантов метрик :
\begin{enumerate}
	\item Косинусная метрика(Сosine). Данная метрика является достаточно универсальной. Формула: \begin{equation}
		\frac{\sum_{i=1}^n v1_i v2_i}{\sum_{i=1}^n \sqrt{v1_i^2}\sqrt{v2_i^2}}
	\end{equation} 
	\item Жаккардова метрика(Jaccard). Данная метрика используется для бинаризованных классов. Введем дополнительные обозначения: 
	\begin{itemize}
		\item $M_{00}$ ~--- общее количество индексов $i$ , когда $v1_i = v2_i = 0$.
		\item $M_{01}$ ~--- общее количество индексов $i$ , когда $v1_i = 0,  v2_i = 1$.
		\item $M_{10}$ ~--- общее количество индексов $i$ , когда $v1_i = 1, v2_i = 0$.
		\item $M_{11}$ ~--- общее количество индексов $i$ , когда $v1_i = v2_i = 1$.
	\end{itemize}
	 Формула:
	\begin{equation}
		\frac{M_{01} + M_{10}}{M_{01} + M_{10} + M_{11}}
	\end{equation}
	\item Метрика Dice. Данная метрика используется для бинаризованных классов. Используя обозначения из предыдущего пункта, формулу можно записать следующим образом:
	\begin{equation}
		\frac{M_{01} + M_{10}}{2 * M_{11} + M_{01} + M_{10}}
	\end{equation}
	\item Евклидова метрика. Это наиболее естественная функция расстояния, возникающая в геометрии, отражающая интуитивные свойства расстояния между точками. Формула метрики:
	 \begin{equation}
		\sqrt{\sum_{i=1}^n (v1_n - v2_n)^2 }
	\end{equation} 
	\item Манхэттенская метрика.  Формула данной метрики очень проста: 
	 \begin{equation}
		\sqrt{\sum_{i=1}^n | v1_n - v2_n | }
	\end{equation} 
	\item Метрика Чебышева. Формула:
	 \begin{equation}
		\sqrt{\max_{i=1}^n | v1_n - v2_n | }
	\end{equation} 
\end{enumerate} 

\begin{table}[H]
	\centering
	\caption{Результаты использования различных метрик расстояния}\label{IP:table:1}
\begin{tabular}{| l |l| l| l| l|}
	\hline
	Метрика & DCG@30 & HUR@5 & IOU@30 &  NAP@30 \\
	\hline
	Cosine & 0.03754 & 0.12851 & 0.02505 & 0.01080 \\
	\hline
	Jaccard &  0.03101 &0.11245 & 0.02155 & 0.00779\\
	\hline
	Dice & 0.03101 & 0.11245 & 0.02155 & 0.00779 \\
	\hline
	Eucledian & 0.03459 & 0.11245 & 0.02302 & 0.00972 \\
	\hline
	Manhattan & 0.03613 & 0.12249 & 0.02377 & 0.01029 \\
	\hline
	Chebyshev & 0.03200 & 0.10843 & 0.02113 & 0.00897 \\
	\hline
\end{tabular}
\end{table}
Как можно заметить, наилучшие показатели модель получила при работе с косинусной метрикой, которая, является и наиболее универсальной. Так как универсальность крайне важна для выполняемой задачи, то данная метрика и будет использоваться в дальнейшем.
 
В случае рекомендаций товаров на основе истории взаимодействий пользователя применяется алгоритм \ref{IP:alg:1}. Здесь $N$ ~--- количество необходимых рекомендаций, $u$ ~--- пользователь, для которого даются рекомендации, $k$ ~--- максимальное число похожих на данный товар товаров, $D_u$ ~--- набор товаров, с которыми взаимодействовал пользователь.   \\
\begin{algorithm}[H]\label{IP:alg:1}
	\caption{Процесс генерации рекомендаций с помощью IP.}
	\KwIn{$N, D_u, z$}
	\KwOut{Набор товаров}
	result = \{\} \\
	i = 0 \\
	sort $D_u$ by popularity\\
	\While{len(result) < N}{
		result $\leftarrow$ result $\wedge$ kNN($D_u[i]$)
	}
	return result
\end{algorithm}  
\ \\

Если алгоритм не позволяет найти необходимое количество товаров ($len(result) < N$), что происходит, например, в ситуации когда пользователь малоактивен и/или начал пользоваться сайтом недавно, результат дополняется результатами модели TopN до необходимого количества рекомендаций. 
Правило брать не более чем k похожих на текущий товара выведено экспериментально, в противном случае рекомендация получается маловариативной, и как соответствие, зачастую худшей по качеству. Число k=3 выведено экспериментальным путем. 
\subsection{Модель UIP}
Модель UIP(User Item Profiles) является гибридной моделью, сочетающей в себе черты моделей с подходами на основе содержимого и моделей коллаборативной фильтрации.

Аналогично предыдушей модели, построенные в представлении данных UIP профили пользователей используются как обучающий датасет для модели sklearn.NearestNeighbours \cite{IP:a1}. Данная модель позволяет находить пользователей, похожих на данного, и, соответственно, рекомендовать пользователю товары, являющиеся популярными у похожих пользователей. Используется представление данных UIP.
Применяется алгоритм, приведенный в листинге \ref{UIP:1}:

\begin{algorithm}[H]\label{UIP:1}
	\caption{Алгоритм генерации рекомендаций с помощью UIP}
	\KwIn{$N, D_u, z$}
	\KwOut{Набор товаров}
	result = \{\} \\
	i = 0 \\
	sort $D_u$ by popularity\\
	\While{len(result) < N}{
		result $\leftarrow$ result $\wedge$ kNN($D_u[i]$)
	}
	return result
\end{algorithm}
\ \\

Аналогично, правило брать не более чем 5 товаров у похожего пользователя выведено экспериментально.
\subsection{Модель DRN}
Данная модель примечательна тем, что она так же, как и модель UIP является гибридной на базе моделей с подходами на основе содержимого и моделей коллаборативной фильтрации \cite{DRN:a1}. Также она примечательна и своей конструкцией ~--- это сиамская нейронная сеть, имеющая структуру, показанную на рисунке \ref{DRN:pic:1}. Задача этой сети ~--- обучиться отличать вещи, которые могут понравиться пользователю. Необычен и процесс обучения ~--- на вход данной модели подаются тройки вида (профиль пользователя, профиль понравившейся ему вещи, профиль не понравившейся ему вещи). 

Снова рассмотрим представление IP, которое суть метрическое пространство $\mathbb{B}^n$ c набором векторов $x \in \mathbb{B}^n$. Обращаясь же к представлению данных UIP, вспомним, что профиль каждого пользователя представляет собой алгебраическую сумму профилей товаров, с которыми он взаимодействовал, то есть это пространство ~--- метрическое пространство $\mathbb{R}^n$ c набором векторов $x \in \mathbb{R}^n$.

На вход данной модели необходимо подать тройку(пользователь, товар с которым он взаимодействовал, товар с  которым он не взаимодействовал). Здесь так же используется аксиома, что если человек взаимодействовал с товаром, то данный товар ему понравился. При этом, несмотря на то, что в данной системе размерности профилей пользователей и товаров идентичны, это не обязательно должно быть так. Так, допустим, в случае добавления в систему механизмов отслеживания поведения каждого пользователя, представляется возможным добавить в профиль пользователя новые данные и/или заменить профиль пользователя. 

Приведем граф модели ~--- рисунок \ref{DRN:pic:1}. Граф использует переменную emb\_dim. Значение данной переменной экспериментально подобрано как 32.

\begin{figure}[H]
	\begin{tikzpicture}[main/.style = {draw, rectangle},node distance= 1cm,align=left] 
		\node[main] (1) {input(,item\_dim)(Pos.)}; 
		\node[main] (2) [right=0.5cm of 1] {input(,item\_dim)(Neg.)};
		\node[main] (3) [right=0.5cm of 2] {input(,user\_dim)};
		\node[main] (4) [below left=0.5cm and -1cm of 2] {Dense(emb\_dim * 2)}; 
		\node[main] (5) [below=0.5cm of 4] {elu}; 
		\node[main] (6) [below=0.5cm of 5] {Dense(emb\_dim )}; 
		\node[main] (7) [below=0.5cm of 6] {elu}; 
		\node[main] (8) [below=3cm of 3] {Dense(emb\_dim )}; 
		\node[main] (9) [below=0.5cm of 8] {elu}; 
		\node[main] (10) [below=0.5cm  of 9] {ScoreLayer(,emb\_dim)}; 
		\node[main] (11) [below=0.5cm  of 7] {ScoreLayer(,emb\_dim)}; 
		\node[main] (12) [below=7cm  of 2] {TripetLossLayer(,emb\_dim)}; 
		%\node[main] (5) [right=1cm of 1] {prepare\_item\_profiles(5)}; 
		%\node[main] (6) [below right=0.5cm and -2cm of 4] {prepare\_user\_profiles(6)}; 
		\draw[->] (1) -- (4);
		\draw[->] (2) -- (4);
		\draw[->] (5) -- (6);
		\draw[->] (6) -- (7);
		\draw[->] (4) -- (5);
		\draw[->] (3) -- (8);
		\draw[->] (8) -- (9);
		\draw[->] (11) -- (12);
		\draw[->] (10) -- (12);
		\draw[->] (7) -- (11);
		\draw[->] (8) -- (11);
		\draw[->] (6) -- (10);
		\draw[->] (9) -- (10);
	\end{tikzpicture}
	\caption{Граф пайплайна системы}\label{DRN:pic:1}
\end{figure}

Конкретизируем значение данного графа.
Данная модель использует две подмодели ~--- для создания эмбеддингов(англ. Embedding) товаров (полносвязный слой(англ. Dense Layer) размерности 64, eLU, полносвязный слой размерности 32, ReLU) и пользователей (полносвязный слой размерности 32, eLU). 

Стоит отметить, что слои TripletLossLayer и ScoreLayer реализованы специально для этой модели. Рассмотрим их подробнее.

Score layer  ~--- слой, который определяет близость эмбеддинга пользователя к эмбеддингу товара. Очевидно, что задача сети ~---  максимизировать близость(минимизировать расстояние) между эмбеддингом пользователя и эмбеддингом товара c которым он взаимодействовал, и, соответственно, максимизировать расстояние между эмбеддингом пользователя и эмбеддингом товара c которым он не взаимодействовал.

Triplet loss layer ~--- слой, который определяет насколько хорошо сеть обучилась. Он использует следующую функцию:
\begin{equation}
	Loss = 1.0 - \frac{1}{1 + e^{-(p\_score-n\_score)}}
\end{equation}
Здесь $p\_score$ ~--- выход Score layer, отвечающего за товары, с которыми пользователь взаимодействовал, $n\_score$ - выход Score layer, отвечающего за товары, с которыми пользователь не  взаимодействовал. 

Из-за сложности и специализированных слоев данная модель имеет низкую скорость обучения и не показывает высоких результатов.

\subsection{Сравнительный анализ моделей}
Для сравнительного анализа моделей была рассчитаны метрики для всех моделей на тестовом датасете, содержащем данные за 2 недели, и сведены в единую таблицу \ref{semifin:table:1}.
\begin{table}[H]
	\centering
	\caption{Сравнительные характеристики моделей}\label{semifin:table:1}
\begin{tabular}{| l |l| l| l| l|}
	\hline
	Модель & DCG@30 & HUR@5 & IOU@30 &  NAP@30 \\
	\hline
	Top30 & 0.02464 & 0.11504 & 0.02029 & 0.00535 \\
	\hline
	ALS & 0.02949 & 0.10442 & 0.01994 &  0.0861 \\
	\hline
	BPR & 0 & 0 & 0& 0\\
	\hline
	AE & 0.07552 & 0.25841 & 0.04696 &  0.02894 \\
	\hline
	IP & 0.03612 & 0.12249 & 0.02375 &  0.01038 \\
	\hline
	UIP & 0.5002 & 0.26707 & 0.0207 &  0.02035 \\
	\hline
	DRN & 0.00861 & 0.03363 & 0.0027 & 0.00401 \\
	\hline
\end{tabular}
\end{table}
Основополагающей метрикой здесь является метрика HappyUsersRatio. Как мы видим, наилучший результат показали модели AE и UIP.

%=======================

\section{Создание ансамбля моделей}
Для улучшения результатов был создан автоматический ансамбль, который выбирает лучшую модель на основе коллаборативной фильтрации и лучшую модель на основе содержимого(в данном случае единственную ~--- IP).  Вследствие того, что в ансамбле присутствуют модели, сходные по характеристикам, а в результате обучения метрики немного меняются в зависимости от данных, этот подход позволяет потенциально всегда выбирать наилучшие модели для заданных магазинов.
Введем $Prods$ ~--- множество всех товаров, доступных в интернет магазине ~--- и рассмотрим алгоритм формирования финальных рекомендаций(листинг \ref{ens:ls:1}). 

Сравним метрики, полученные таким способом, с уже имеющимися и занесем данные в таблицу \ref{fin:table:1}.

\begin{table}[H]
	\centering
	\caption{Сравнительные характеристики моделей и ансамбля}\label{fin:table:1}

\begin{tabular}{| l |l| l| l| l|}
	\hline
	Модель & DCG@30 & HUR@5 & IOU@30 &  NAP@30 \\
	\hline
	Top30 & 0.02464 & 0.11504 & 0.02029 & 0.00535 \\
	\hline
	ALS & 0.02949 & 0.10442 & 0.01994 &  0.0861 \\
	\hline
	BPR & 0 & 0 & 0& 0\\
	\hline
	AE & 0.07552 & 0.25841 & 0.04696 &  0.02894 \\
	\hline
	IP & 0.03612 & 0.12249 & 0.02375 &  0.01038 \\
	\hline
	UIP & 0.5002 & 0.26707 & 0.0207 &  0.02035 \\
	\hline
	DRN & 0.00861 & 0.03363 & 0.0027 & 0.00401 \\
	\hline
	Ансамбль & 0.8889 & 0.29707 & 0.04164 &  0.02504 \\
	\hline
\end{tabular}
\end{table}
Следует отметить, что модели AE и UIP имеют одни из лучших метрик, а итоговый ансамбль опережает даже их. Таким образом, при использовании предложенного кастомного ансамбля моделей мы получаем несколько лучшие результаты без проигрыша во времени выполнения, что говорит о целесообразности применения ансамблей моделей по сравнению с отдельными моделями. %(Указать в процентах )

\begin{algorithm}[H]\label{ens:ls:1}
	\caption{Алгоритм формирования финальных рекомендаций.}
	\KwIn{\\
		$CB\_Recs$ - рекомендации модели на основе содержимого. $K$ или менее элементов $\in Prods$;  \\
		$CF\_Recs$ - рекомендации модели коллаборативной фильтрации. $K$ или менее элементов $\in Prods$;\\
		$TP\_Recs$ - рекомендации TopK модели. $K$ элементов $\in Prods$;\\
		$K$ - количество рекомендаций.
	}
	\KwOut{Out\_Recs - финальные рекомендации.  $K$ элементов $\in Prods$}
	$T \leftarrow  CB\_Recs  \cap  CF\_Recs; $\\
	$Out\_Recs \leftarrow T$;\\
	i,j $\leftarrow$ 0, $sw \leftarrow$ True;\\
	\While{len(Out\_Recs) < K}
	{
		\eIf{sw}
		{
			\If{CB\_Recs[i] $\notin$ Out\_Recs}
			{$Out\_Recs$.append($CB\_Recs[i]$);}
			i++;
		}
		{
			\If{CF\_Recs[j] $\notin$ Out\_Recs}
			{$Out\_Recs$.append($CF\_Recs[j]$);}
			j++;
		}
		$sw$ $\leftarrow$ $\hat{sw}$
	}
	\If{len(Out\_Recs) < K}
	{
		$Out\_Recs$.join($TP\_Recs$[K-$len(Out\_Recs)$:])
	}
	
\end{algorithm}

%Переместить алгоритм пониже. Добавить какую-то завершающую фразу(общий вывод из пункта)

%=======================

%=======================
\newpage
\addcontentsline{toc}{section}{Заключение}
\section*{Заключение}

На данный момент создана универсальная рекомендательная система с адаптивным ансамблем моделей, способная к полностью автоматическому переформированию представлений данных, пересозданию и переобучению моделей, приспособленная для любого интернет-магазина при условии предоставления им стандартизированной выгрузки данных. На данный момент производится А/B тестирование на серверах интернет-магазинов для определения промышленных показателей эффективности системы. 

%Добавить то же самое, что во введении. 2-3 полных абзаца.


Полученные на настоящий момент результаты работы были опубликованы на конференции МКО-2022 в виде тезисов и видеопрезентации доклада [Приложение 2], а также на конференции СИТО-2022 в виде тезисов и презентации доклада.
%Должно 


%=======================
\newpage

\addcontentsline{toc}{section}{Литература}
\renewcommand{\refname}{\centering \textbf{Литература}}

% БИБЛИОГРАФИЯ

\begin{thebibliography}{0}
	
\bibitem{INTRO:a1}
{
	Официальный сайт McKinsey. ~--- URL: \url{https://www.mckinsey.com/industries/retail/our-insights/how-retailers-can-keep-up-with-consumers} (дата обр. 20.05.2022)
}

\bibitem{stud:kimfalk1}{
	\textit{Фальк, К}. Рекомендательные системы на практике / пер. с англ. Д. М. Павлова. – М.:
	ДМК Пресс, 2020. – 448 с.
}

\bibitem{implicit:all}{
	Documentation of "implicit" \ library ~--- \url{https://benfred.github.io/implicit/} (дата обр. 10.05.2022)
}

\bibitem{ALS:CFIFD}{
	\textit{Hu, Y., Koren, Y. , Volinsky, C.}
	Collaborative Filtering for Implicit Feedback Datasets. // Proceedings ~--- IEEE International Conference on Data Mining, ICDM 2008. 263-272.
}

\bibitem{ALS:rwe}{
	\textit{Takács, G., Tikk, D.}
	Alternating least squares for personalized ranking. ~--- 2012. ~--- URL: \url{https://www.researchgate.net/publication/254464370_Alternating_least_squares_for_personalized_ranking} (дата обр. 2.05.2022)
}

\bibitem{ALS:recsys}{
	\textit{Takács, G., Pilászy, I., Tikk, D. }
	Applications of the conjugate gradient method for implicit feedback collaborative filtering.// Proceedings ~--- 5th ACM conference on Recommender systems, 2011. Association for Computing Machinery, New York, NY, USA, 297–300.
}

\bibitem{BPR:1205}{
	\textit{Rendle, S., Freudenthaler, C., Ganther Z., Schmidt-Thieme, L.}
	BPR: Bayesian Personalized Ranking from Implicit Feedback. ~--- 2012. ~--- URL:\url{https://arxiv.org/pdf/1205.2618.pdf} (дата обр. 15.05.2022)
}

\bibitem{AE:a1}{
	\textit{Kuchaiev, O., Ginsburg, B.},
	Training Deep AutoEncoders for Collaborative Filtering. ~--- 2017. ~---
	URL: \url{https://arxiv.org/abs/1708.01715 } (дата обр. 14.05.2022)
}

\bibitem{IP:a1}
{
	Documentation of "scikit-learn" \ library, Nearest Neighbour module.
	URL: \url{https://scikit-learn.org/stable/modules/neighbors.html} (дата обр. 11.05.2022)
}

\bibitem{IP:a2}
{
	\textit{Maneewongvatana, S., Mount, D. M.},
	Analysis of approximate nearest neighbor searching with clustered point sets, ~--- 1999 ~---
	URL: \url{https://arxiv.org/abs/cs/9901013} (дата обр. 10.04.2022)
}

\bibitem{DRN:a1}{
	\textit{Elkahky A., Song, Y., He, X.},
	A Multi-View Deep Learning Approach for Cross Domain User Modeling in Recommendation Systems. //
	Proceedings ~--- 24th International Conference on World Wide Web, 2015. International World Wide Web Conferences Steering Committee, Republic and Canton of Geneva, CHE, 278–288. URL: \url{https://doi.org/10.1145/2736277.2741667} (дата обр. 20.05.2022)
	
}

\end{thebibliography}

\newpage
\addcontentsline{toc}{section}{Приложения}
\renewcommand{\thesection}{\arabic{section}}
\renewcommand{\thesubsection}{\arabic{subsection}}
%Переоформить как одно приложение.

\subsection{Приложение 1. Ссылка на код проекта.}
\url{https://gitlab.visionsystems.tech/recommendation}
\subsection{Приложение 2. Ссылка на видеопрезентацию МКО-2022.}
\url{https://drive.google.com/file/d/1MRJck_qHvx7F4EjwNzrFtUY8x-wwdHqV/view?usp=sharing}

\end{document}
% ----------------------------------------------------------------


\lstset{ %
language=Python,                 % выбор языка для подсветки (здесь это С++)
basicstyle=\small\sffamily, % размер и начертание шрифта для подсветки кода
numbers=left,               % где поставить нумерацию строк (слева\справа)
numberstyle=\tiny,           % размер шрифта для номеров строк
stepnumber=1,                   % размер шага между двумя номерами строк
numbersep=5pt,                % как далеко отстоят номера строк от подсвечиваемого кода
backgroundcolor=\color{white}, % цвет фона подсветки - используем \usepackage{color}
showspaces=false,            % показывать или нет пробелы специальными отступами
showstringspaces=false,      % показывать или нет пробелы в строках
showtabs=false,             % показывать или нет табуляцию в строках
frame=single,              % рисовать рамку вокруг кода
tabsize=2,                 % размер табуляции по умолчанию равен 2 пробелам
captionpos=t,              % позиция заголовка вверху [t] или внизу [b]
breaklines=true,           % автоматически переносить строки (да\нет)
breakatwhitespace=false, % переносить строки только если есть пробел
escapeinside={\%*}{*)}   % если нужно добавить комментарии в коде
extendedchars=true,
commentstyle=\color{mygreen},    % comment style
stringstyle=\bf,
commentstyle=\ttfamily\itshape,
keepspaces=true % пробелы между русскими буквами
aboveskip=3mm,
belowskip=3mm

}


\renewcommand\NAT@bibsetnum[1]{\settowidth\labelwidth{\@biblabel{#1}}%
   \setlength{\leftmargin}{\bibindent}\addtolength{\leftmargin}{\dimexpr\labelwidth+\labelsep\relax}%
   \setlength{\itemindent}{-\bibindent+\fivecharsapprox}%
   \setlength{\listparindent}{\itemindent}
\setlength{\itemsep}{\bibsep}\setlength{\parsep}{\z@}%
   \ifNAT@openbib
     \addtolength{\leftmargin}{\bibindent}%
     \setlength{\itemindent}{-\bibindent}%
     \setlength{\listparindent}{\itemindent}%
     \setlength{\parsep}{0pt}%
   \fi
}
\renewcommand{\thesection}{\arabic{section}.}
\renewcommand{\thesubsection}{\arabic{section}.\arabic{subsection}.}
